%% LyX 2.0.6 created this file.  For more info, see http://www.lyx.org/.
%% Do not edit unless you really know what you are doing.
\documentclass[english]{article}
\usepackage[T1]{fontenc}
\usepackage[utf8]{luainputenc}
\usepackage{array}

\makeatletter

%%%%%%%%%%%%%%%%%%%%%%%%%%%%%% LyX specific LaTeX commands.
%% Because html converters don't know tabularnewline
\providecommand{\tabularnewline}{\\}

\makeatother

\usepackage{babel}
\begin{document}
{\footnotesize{}}%
\setlength{\tabcolsep}{0.0em}
{\renewcommand{\arraystretch}{0.5}
\begin{tabular}{|>{\centering}p{2cm}|>{\raggedright}p{5cm}|>{\centering}p{4cc}|}
\hline 
{\footnotesize{Refactoring}} & {\footnotesize{Move Class}} & {\footnotesize{P.C?}}\tabularnewline
\hline 
\hline 
{\footnotesize{Code Package}} & {\footnotesize{Move an specific ClassUnti from a source Package to
a target Package}} & {\footnotesize{Yes}}\tabularnewline
\hline 
{\footnotesize{Structure Package}} & {\footnotesize{If the target Package is associated to an architectural
elements by means of the association implementation the value of meta-attribute
named density should be propagated }} & {\footnotesize{Yes}}\tabularnewline
\hline 
{\footnotesize{Data Package}} & {\footnotesize{Not applied}} & {\footnotesize{Not applied}}\tabularnewline
\hline 
{\footnotesize{Conceptual Package}} & {\footnotesize{If the moved class is associated to any conceptual
elements by means of the association implementation this conceptual
elements should be moved to a correspondent associated element of
the target Package. }} & {\footnotesize{Yes}}\tabularnewline
\hline 
\end{tabular}}
\end{document}
