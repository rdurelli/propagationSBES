%!TEX root = /Users/rafaeldurelli/Dropbox/Artigos Elaborados/KDM propagation_2015/sbes_2015_kdm_propagation/sbes2015_kdm_propagation.tex




In~\cite{4440135}, Enrico Biermann et al. propose to use the Eclipse Modeling Framework (EMF), a modeling and code generation framework for Eclipse applications based on structured data models. They introduce the EMF model refactoring by defining a transformation rules applied on EMF models. EMF transformation rules can be translated to corresponding graph transformation rules. If the resulting EMF model is consistent, the corresponding result graph is equivalent and can be used for validating EMF model refactoring. Authors offer a help for developer to decide which refactoring is most suitable for a given model and why, by analyzing the conflicts and dependencies of refactorings. This initiative is closed to the model driven architecture (MDA) paradigm~\cite{Kleppe:2003} since it starts from the EMF metamodel applying a transformation rules.

In~\cite{Rui:2003} Rui, K. and Butler, apply refactoring on use case models, they propose a generic refactoring based on use case metamodel. This metamodel allows creating several categories of use case refactorings, they extend the code refactoring to define a set of use case refactorings primitive. This refactoring is very specific since it is focused only on use case model, the issue of generic refactoring is not addressed, and these works do not follow the MDA approach.

Another work on model refactoring is proposed in~\cite{Zhang05genericand}, based on the Constraint-Specification Aspect Weaver (C-SAW), a model transformation engine which describes the binding and parameterization of strategies to specific entities in a model. Authors propose a model refactoring browser within the model transformation engine to enable the automation and customization of various refactoring methods for either generic models or domain-specific models. The transformation proposed in this work is not based on any metamodel, it is not an MDA approach.


In the line of language independent refactoring and metamodelling, Sander et al.~\cite{Tichelaar00}, study the similarities between refactorings for Smalltalk and Java, and build the FAMIX model. It provides a language-independent representation of object- oriented source code. It is an entity-relationship model that models object-oriented source code at the program entity level, with a tool to assist refactoring named MOOSE. FAMIX does not take account neither complex features in strongly typed languages, nor aspects of advanced inheritance and genericity. This approach is not really independent from language since the refactoring transformation is achieved directly on the original code. This alternative forces to implement transformers of specific code for each language. These code transformers use an approach based on text using regular expressions.



