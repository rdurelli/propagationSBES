%!TEX root = /Users/rafaeldurelli/Dropbox/Artigos Elaborados/KDM propagation_2015/sbes_2015_kdm_propagation/sbes2015_kdm_propagation.tex
\section{Change Propagation in KDM} % (fold)
\label{sec:motivation_and_running_example}

In our previous work~\cite{IRIDurelliCatalogo}, we introduced a refactoring catalogue for KDM for managing evolution of a software system. This catalogue served as a starting point to investigate how the changes affect the KDM's levels. 

Considering the system described earlier it is possible to remark some problem or even to add new requirements that will propagate changes at KDM's levels. For instance, a problem that can be noticed is that both classes \texttt{Student} and \texttt{Instructor} ( see figure~\ref{fig:system} \ding{182}) should be contained in \texttt{Model} package not in \texttt{GUI} package, respectively. One way to fix this would be to apply the refactoring \textit{Move Class}. Then both classes should be moved to the correct package.
%
%
%
%
Regarding to a new requirement let's pretend someone has identified that the class 	\texttt{Student} is doing work that should be done by two classes, e.g., it contains attributes that hold informations upon student's addresses. In order to fulfill this new requirement one should apply the refactoring \textit{Extract Class}. Then a new class named Address would be created (which is a POJO and also an ORM) and all student's attributes related to address would be moved to this new class.

%The action of this refactoring should propagate throughout other KDM's levels, such as the data level\footnote{The KDM's level that contains information on data base schema, table, column, primary key, etc}. 
However, these changes would rise a synchronization problem among all KDM's levels/packages. For instance, in both described refactoring it is necessary a skilled domain expert into KDM to identify all the metaclasses in the system which involve/reference the classes aforementioned and correct them respectively in all KDM packages, i.e., propagate all refactoring's impact throughout all KDM's packages/level. 

In the matter of the refactoring \textit{Move Class} (move \texttt{Student} and \texttt{Instructor} from \texttt{GUI} package to \texttt{Model} package) changes should be propagated to the \texttt{Structure Package} and to the \texttt{Conceptual Package} to maintain the model synchronized. Regarding the \texttt{Structure Package}, the \texttt{density}, i.e., aggregation relation ship between the layer \texttt{View} and the layer \texttt{Controller} would change from 4 to 2 - once the primitives relationships \texttt{Create} and \texttt{Extends} would no longer exist from the package \texttt{GUI} to the package \texttt{CTR}. On the other hand, the resulting of this refactoring would update the density between the layer \texttt{Model} and \texttt{Controller}, instead of 2 it should be 4, as \texttt{Creates} and \texttt{Extends} were also moved along with its classes, \texttt{Student} and \texttt{Instructor}. Concerning to the \texttt{Conceptual Package}, the  RuleUnit\_1.1 that is associated with \texttt{Instructor} should also be moved to corresponding scenario, i.e, the scenario that is associated with the package that contains now the class \texttt{Instructor} - ScenarioUnit\_3. 

About the refactoring \textit{Extract Class}, the extracted class \texttt{Address} would be a POJO (it would be contained in \texttt{Model package} and it would also be an ORM - therefore, the action of this refactoring should be propagated throughout  the \texttt{Data package}, i.e., the instance of \texttt{Address} should be associated with a metaclass \texttt{RelationalTable}, and its attributes should be associated with  of \texttt{ColumnSet}.

%, i.e., the relationship among the layers should be propagated automatically. Similarly, considering the refactoring \textit{Extract Class}, where a new POJO and ORM class is created, the data's level also should be propagated.

These propagation seen to be easy to apply, however, in a complex system comprising all kdm's packages/levels, propagate all changes after a refactoring is a difficult and error-prone task. Even identifying the affected parts of the KDM's packages/levels is not an easy and straightforward process. 

%-------------------------
In the context of model-driven  refactoring, if any change occurs at any KDM's subtree the change should be propagated to other elements.
%
%These levels indicate problems related to KDM propagation of changes. 
%
For instance, when the elements of \texttt{CodeModel} suffer any kind of changes (e.g., are refactored), its instances, i.e., \texttt{ClassUnits}, \texttt{MethodUnits}, \texttt{StorableUnits}, etc, and related elements must be adapted accordingly so that their validity and correctness is preserved respectively. In addition, if we want to preserve others parts of KDM, like the system's structure and the business rules the  \texttt{StructureModel} and \texttt{ConceptualModel} also need to adapt, respectively. %What is more, as we have mentioned, in practice there are usually at least one instance of each KDM's model applied in a single system, e.g., the system architecture conforms to Structure Model, the source-code conforms to Code Model, etc. 
In general, a change at one KDM's package/level should trigger a cascade of changes at other models. We call such sequences of adaptations change propagation.

As we can see in Figure~\ref{fig:allKDMLayers}, there are not only horizontally relations between the models, but the elements of the system can also be vertical related across the vertical partitions. A few examples are denoted by the red/blue dashed arrows. For instance, there is a relation between a \texttt{CodeModel} (its respective metaclasses) with the \texttt{StructureModel} - which means that a change in one of the ends of the relation can influences the other.

Considering these KDM's models as individual artifacts lead up to refactoring of each affected model separately, causing synchronization problem among them. However, this is an error-prone solution since we need to apply a refactoring at any KDM's models and then propagate all change in order to keep a KDM instance synchronized. %a domain expert who is able to identify all the affected models and propagate the changes. 
%In this context, we present an approach to detect and propagate all the changes throughout all KDM's levels after one to apply a refactoring 



%-------------------------

In order to fulfill this limitation and create an automatized process named Propagation-Aware Refactorings (PARef) that contains three main steps. The first step is the identification of all dependent elements related to a specific refactoring. In the second step the refactoring of all identified elements are performed using a model-to-model transformation language - the third step is the propagation of changes in order to keep all the dependent models synchronized. %In the following sections we show in detail that change propagation in KDM is a complex process that can be solved semi-automatically and, hence, efficiently and precisely if we provide a rigorous theoretical background. 
In the following sections our approach is presented.
 
