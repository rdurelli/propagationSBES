 Figure~\ref{fig:domain_engineering} depicts a simplified set of activities that must be done at the DE phase. 
 A more detailed set of activities is shown in another paper \cite{gottardiwasp2011}.
 During the first activity (marked with `A'), the engineer should get enough expertise related to 
 the domain prior to developing all necessaries artifacts. 
 Afterwards, the CFF has to be elaborated itself (`B'), consequently all classes, 
 interfaces and aspects will be developed by the domain engineer. 
 At the third activity, the feature model has to be modeled (`C'), in which describes all features mandatory, 
 optional and alternative of the CFF. 
 The Requirements Reuse Model (RRM) is the last artifact that has to be modeled (`D').
 This model it is important to document the CFF and to represent its reuse requirements. 

\begin{figure}[!h]
\centering
  % Requires \usepackage{graphicx}
 \includegraphics[width=0.32\textwidth, height=0.14\textheight]{figuras/domain_eng_diag.\figext}
\caption{Domain Engineering Activity Diagram\label{fig:domain_engineering}}
\end{figure} 
 
 The last activity that has to be done is the uploading of the CFF (`E'),	 feature model and RRM.   
 An example of a CFF being uploaded is shown in Figure~\ref{fig:uploadedcffs}, wherein 
 all information related to the CFF must be filled in, besides the information, 
 the feature model, RRM and the CFF itself must be uploaded, as well.
 These information and artifacts are available at the AE phase in order to assist the application engineer identify a 
 CFF that may be used to fulfill the application requirements that will be developed.
 Further details upon this phase is presented in Section~\ref{sec:domain_engineering_example}.   
 
%\begin{figure}[!h]
%\centering
%\includegraphics[width=\figwidth]{figuras/UploadExample.\figext}
%\caption{Uploading a Persistence CFF in the Repository\label{fig:uploadedcffs}}
%\end{figure}
