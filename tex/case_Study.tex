%!TEX root = /Users/rafaeldurelli/Dropbox/Artigos Elaborados/KDM propagation_2015/sbes_2015_kdm_propagation/sbes2015_kdm_propagation.tex

\section{Case Study}

In this section we present a case study showing that our approach can be used to support the change propagation in KDM models. We have used a real-life legacy information system. 
Notice that the case study was carried out following the protocol for planning, conducting and reporting case studies proposed by Brereton et al. in [17] improving the rigor and validity of the study. The next subsections show more details about the main phase defined in this protocol, such as: background, design, case selection, case study procedure, data collection, analysis and interpretation and validity evaluation.

\subsection{Research Question}

According to the protocol proposed by Brereton et al. in~\cite{Brereton:2008} firstly it is needed to identify previous research on the topic. Hence, in Section~\ref{sec:related_work} we stated some researches related to refactoring in models. However, not of them are using ADM/KDM. We are particularly focus on the propagation of changing in different views of a KDM model. In this context, the object of this study is the proposed approach, and the purpose of this study is the evaluation of the approach herein described related to its effectiveness and efficiency.

Therefore, taking into account the object and purpose of the study, it was defined one research question, as follows:

\begin{itemize}
\item \textbf{RQ$_1$}: Given a set of refactoring, can the proposed approach propagate all the changes effectively throughout all KDM levels?
\end{itemize}


\subsection{Design}

The described case study consist of a single case~\cite{Brereton:2008}. It was focused on a single legacy system. To assess the effectiveness of the proposed approach through the \textbf{RQ$_1$}, we use some oracles. As each refactoring has its own characteristics and modifies specific model elements, it is possible to predict all the expected changes in other KDM levels. So, considering our set of developed refactorings, we had to develop some oracles for each refactoring. The complete oracle can be see at www.mudar.com.br.

\subsection{Case Selection}

In this section is described the suitable case that was chosen to be studied. Some criteria were applied to select the suitable case, as follows: (i) it must be an enterprise system, (ii) it must be a Java-based system, (iii) it must be a legacy system and (iv) it must be of a size not less than 10 KLOC. After applying these criteria we chose LabSys (Laboratory System) it is currently used by Federal University of Tocantins (UFT). It is used to control the use of laboratories in the entire university. 

\subsection{Case Study Procedure}\label{sec:caseStudyProcedure}

In this section is shown how the execution of the study was planned. Notice that the execution was aided by the tool developed to support the proposed approach. The case study was carried out in a machine with an Intel Core I5 CPU 2.5GHz, 8GB of physical memory running Mac OS X 10.8.4.

The proposed approach uses as initial artefact a KDM instance,. Therefore, firstly we adopted a reverse engineering to  transform the LabSystem source-code into a KDM instance to apply our approach. In this step we have used MoDisco\cite{Brunele20141012}, which is a parser that get as input java source-code and then return as output a KDM instance. Currently, MoDisco only generates the KDM \texttt{Code package}, other KDM packages are extremely important to evaluate our approach. Therefore, we have manually instantiated the followings KDM packages: \texttt{Structure Package}, \texttt{Data Package}, and \texttt{Conceptual Package}.


We selected for refactorings for our evaluation: \textit{Extract Class}, \textit{Move Class}, \textit{Extract Layer} and \textit{Remove Class}. %We applied each of the four refactorings to every possible location in KDM instance. 
It is worth to notice that all refactorings were applied completely automatically by means of our devised proof-of-concept tool. To deal with refactorings that go into infinite loops, we set three minutes timeout interval. More specifically, we applied the \textit{Extract Class} to one class that had more than 300 LOC (Line of Code); we applied the \textit{Move Class} to a set of  class from a package to another package; we applied the \textit{Extract Layer} to a layer that contains at least 20 classes; finally we applied the \textit{Remove Class} to a class that contained at least 15 primitive relationships. After applied all refactorings we verify whether them were successful, i.e., if the intended refactoring could be performed, and if all the expected propagations were generated on the model. 

\subsection{Data Collection and Interpretation}

We verify, based on a set of oracle, whether all refactoring were successfully propagated throughout all KDM models. By using these information gathered we can draw conclusion and answer the \textbf{RQ$_1$}.

\begin{table}
\caption{Propagation - Extract Class}
\label{table:propagationExtractClass}
{\footnotesize{}}%
\setlength{\tabcolsep}{0.0em}
{\renewcommand{\arraystretch}{0.5}
\begin{tabular}{|>{\centering}p{2cm}|>{\raggedright}m{5cm}|>{\centering}p{4cc}|}
\hline 
{\footnotesize{Refactoring}} & {\footnotesize{Extract Class}} & {\footnotesize{P. C?}}\tabularnewline
\hline 
\hline 
\multirow{4}{2cm}{{\footnotesize{Code Package}}} & {\footnotesize{Create an instance of ClassUnit that represent the
new Class}} & {\footnotesize{Yes}}\tabularnewline
\cline{2-3} 
 & {\footnotesize{Move all StorableUnits to the new ClassUnit}} & {\footnotesize{Yes}}\tabularnewline
\cline{2-3} 
 & {\footnotesize{Move all MethodUnit to the new ClassUnit}} & {\footnotesize{Yes}}\tabularnewline
\cline{2-3} 
 & {\footnotesize{Create an intance of HasType, which represent an association
between the new ClassUnit and the old ClassUnit}} & {\footnotesize{Yes}}\tabularnewline
\hline 
{\footnotesize{Structure Package}} & {\footnotesize{Not Applied}} & {\footnotesize{Not Applied}}\tabularnewline
\hline 
\multirow{3}{2cm}{{\footnotesize{Data Package}}} & {\footnotesize{Create a instance of RelationalTable owning the name
of the new ClassUnit}} & {\footnotesize{Yes}}\tabularnewline
\cline{2-3} 
 & {\footnotesize{For each StorableUnit it is necessary to create a ItemUnit,
which represent the RelationaTable columns.}} & {\footnotesize{Yes}}\tabularnewline
\cline{2-3} 
 & {\footnotesize{Create an instance of UniqueKey that represent the
primary key of the RelationalTable.}} & {\footnotesize{Yes}}\tabularnewline
\hline 
{\footnotesize{Conceptual Package}} & {\footnotesize{Not Applied}} & {\footnotesize{Not Applied}}\tabularnewline
\hline 
\end{tabular}}
\end{table}
\begin{table}
\caption{Propagation - Move Class}
{\footnotesize{}}%
\setlength{\tabcolsep}{0.0em}
{\renewcommand{\arraystretch}{0.5}
\begin{tabular}{|>{\centering}p{2cm}|>{\raggedright}p{5cm}|>{\centering}p{4cc}|}
\hline 
{\footnotesize{Refactoring}} & {\footnotesize{Move Class}} & {\footnotesize{P.C?}}\tabularnewline
\hline 
\hline 
{\footnotesize{Code Package}} & {\footnotesize{Move an specific ClassUnti from a source Package to
a target Package}} & {\footnotesize{Yes}}\tabularnewline
\hline 
{\footnotesize{Structure Package}} & {\footnotesize{If the target Package is associated to an architectural
elements by means of the association implementation the value of meta-attribute
named density should be propagated }} & {\footnotesize{Yes}}\tabularnewline
\hline 
{\footnotesize{Data Package}} & {\footnotesize{Not applied}} & {\footnotesize{Not applied}}\tabularnewline
\hline 
{\footnotesize{Conceptual Package}} & {\footnotesize{If the moved class is associated to any conceptual
elements by means of the association implementation this conceptual
elements should be moved to a correspondent associated element of
the target Package. }} & {\footnotesize{Yes}}\tabularnewline
\hline 
\end{tabular}}
\end{table}
\begin{table}
\caption{Propagation - Extract Layer}
{\footnotesize{}}%
\setlength{\tabcolsep}{0.0em}
{\renewcommand{\arraystretch}{0.5}
\begin{tabular}{|>{\centering}p{2cm}|>{\raggedright}p{5cm}|>{\centering}p{4cc}|}
\hline 
{\footnotesize{Refactoring}} & {\footnotesize{Extract Layer}} & {\footnotesize{P.C?}}\tabularnewline
\hline 
\hline 
\multirow{2}{2cm}{{\footnotesize{Code Package}}} & {\footnotesize{Create an instance of Package}} & {\footnotesize{Yes}}\tabularnewline
\cline{2-3} 
 & {\footnotesize{Move a the selected ClassUnit from a Package to the
new Package}} & {\footnotesize{Yes}}\tabularnewline
\hline 
\multirow{4}{2cm}{{\footnotesize{Structure Package}}} & {\footnotesize{Create an instance of Layer }} & {\footnotesize{Yes}}\tabularnewline
\cline{2-3} 
 & {\footnotesize{Create an instance of AggregationRelationship between
the new Layer and the old one}} & {\footnotesize{Yes}}\tabularnewline
\cline{2-3} 
 & {\footnotesize{Associate the new Layer by means of the association
implementation with the new Package}} & {\footnotesize{Yes}}\tabularnewline
\cline{2-3} 
 & {\footnotesize{Summing up all primitive relationship to compute the
meta-attribute density}} & {\footnotesize{Yes}}\tabularnewline
\hline 
{\footnotesize{Data Package}} & {\footnotesize{Not applied}} & {\footnotesize{Not applied}}\tabularnewline
\hline 
{\footnotesize{Conceptual Package}} & {\footnotesize{If the moved ClassUnits are associated to any conceptual
elements by means of the association implementation these conceptual
elements should be moved to a correspondent associated element of
the target Package. }} & {\footnotesize{Yes}}\tabularnewline
\hline 
\end{tabular}}
\end{table}
\begin{table}
\caption{Propagation - Remove Class}
\label{tab:propagationRemoveClass}
{\footnotesize{}}%
\setlength{\tabcolsep}{0.0em}
{\renewcommand{\arraystretch}{0.5}
\begin{tabular}{|>{\centering}p{2cm}|>{\raggedright}p{5cm}|>{\centering}p{4cc}|}
\hline 
{\footnotesize{Refactoring}} & {\footnotesize{Remove Class}} & {\footnotesize{P.C?}}\tabularnewline
\hline 
\hline 
{\footnotesize{Code Package}} & {\footnotesize{Delete the selected instance of a ClassUnit}} & {\footnotesize{Yes}}\tabularnewline
\hline 
{\footnotesize{Structure Package}} & {\footnotesize{If the removed ClassUnit was contained into a specific
Structure element then summing up all primitive relationship and overwrite
the meta-attribute density}} & {\footnotesize{Yes}}\tabularnewline
\hline 
{\footnotesize{Data Package}} & {\footnotesize{If the removed ClassUnit was associated with an instance
of RelationalTable, then it should also be removed}} & {\footnotesize{Yes}}\tabularnewline
\hline 
{\footnotesize{Conceptual Package}} & {\footnotesize{If the removed ClassUnit is associated to any conceptual
elements by means of the association implementation these conceptual
elements should be removed }} & {\footnotesize{Yes}}\tabularnewline
\hline 
\end{tabular}}
\end{table}


%\begin{table}[h]
%\centering
%\caption{Propagation Result\label{tab:propagation}}
%~~\\
%\begin{tabularx}{
%.30\textwidth}{|c|X|}
%There is no difference between using our tool and using an ad-hoc reuse process in terms of productivity (time) to couple sucessfully a CF with an application.
%\hline \cellcolor[gray]{\shadow} Refactoring & \footnotesize{Propagation Corrected?}
%\\
%\hline \cellcolor[gray]{\shadow} Extract Class & \footnotesize{Yes}
%\\
%\hline \cellcolor[gray]{\shadow} Move Class & \footnotesize{Yes}
%\\
%\hline \cellcolor[gray]{\shadow} Extract Layer & \footnotesize{Yes}
%\\
%\hline \cellcolor[gray]{\shadow} Remove Class & \footnotesize{Yes}
%\\
%\hline
%\end{tabularx}
%\end{table}

Tables~\ref{table:propagationExtractClass} to~\ref{tab:propagationRemoveClass} summaries the results related to each refactoring applied and its respective propagations. ``P.C?" stands for ``Propagation Corrected?" As can be seen the all the changes were effectively propagated throughout all KDM levels. Which means that in this case our approach could automatically execute truly relevant propagation throughout all KDM levels when dealing with the refactorings: \textit{Extract Class}, \textit{Move Class}, \textit{Extract Layer} and \textit{Remove Class}. Thereby, the \textbf{RQ$_1$} can be answered as true, that is, the proposed approach can propagate changes effectively throughout all KDM levels.