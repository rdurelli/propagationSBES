%As stated beforeKDM does not provides suitable meta-classes to apply refactorings. 

Previous work on model refactoring focused on UML models (Biermann et al., 2006; Mens, 2005; Mens et al., 2007). However, when these UML models are refactored, the respective changes have to be propagated into either different artefacts or other levels, i.e., distinct models; otherwise, the system represented in model is no longer consistent with the model. As the KDM is a meta-model that aims to model all of a given system, it allows the propagation and traceability of changes after the application of a refactoring.


Previous works are related to static context - within a static context, refactoring of model will probably never cause any problems. But many models evolve in time: elements are renamed, layer are created, relationship are changed, references are re-targeted and classes are inlined or extracted in order to create or collapse inheritance hierarchies or just to improve the model - in short: the model is refactored. Regarding only UML models, applying these refactorings is well understood and implemented in several ways. However, in the context of the KDM models there is none research in this area. Therefore, we claim that research must to be done in this direction. The problem is usually caused by the strong interconnection between KDM's packages, all KDM's packages are connected somehow (see Section~\ref{sec:background}), e.g., the package Structure uses meta-classes from package Code, the package Code uses meta-classes from package core, etc. Therefore, if any meta-class is just referencing the refactored model meta-class, after refactoring, this reference can be unset. Hence, the KDM model is invalid because of inconsistencies between its packages. 


Therefore in this section we propose an approach to assist the propagation of refactoring into KDM's packages. Our approach starts with Fowler`s definition of each refactoring. As these definitions were introduced for the refactoring of (object oriented) code, we adapt the definitions for KDM model refactorings. Subsequently we investigate how the changes affect the KDM's packages. Herein we consider different aspects concerning the KDM's packages. For instance, depending on the meta-class the refactoring may cause minor or major changes to be propagated into other packages.

In order to explain the propagation of changes in KDM in this section a set of examples is used. The first system is presented in Figure~\ref{fig:system} (A).As noted in Figure~\ref{fig:system} (A) two layers have been defined. The first one is ``View'' and the second layer is ``Controller''. Inside of each layer there is one package. The relationships ``GUI inside View'' and ``CTR inside Controller'' mean that ``GUI'' and ``CTR'' are in container ``View'' and ``Controller'', respectively or in some sub-container of  ``View'' and ``Controller'', transitively. In the same way, the relationship ``StudentGUI inside GUI'' means that ``StudentGUI'' is in container ``GUI'' or in some sub-container of ``GUI''. For relationships, let R' be the corresponding aggregated relationship, which represents the number summing all primitive relationships, i.e., ``Calls'', ``Creates'', ``Extends'', etc. 

The corresponding, though simplified KDM instance of Figure~\ref{fig:system} (A) is depicted in Figure~\ref{fig:system} (B). It illustrates a KDM instance as a UML object diagram for the sake of simplicity, note that this diagram represents the system in  as a tree of nodes containing some KDM`s meta-classes. Analysing both figures it is evident that each element presented in Figure~\ref{fig:system} (A) has a meta-class in KDM to represent it. For instance, the layers are represented in KDM using the meta-class Layer. Each primitive relationship has also a meta-class in KDM. The meta-class ``AggregatedRelationship'' represents a set of primitive KDM relationships. 

In order to increase the modularity of the system consider that the engineer has chosen to apply refactoring extract package to create a package model. In Figure~\ref{fig:atl} is depicted a chunk of code written in ATL responsible to perform the refactoring extract package.

Firstly, an instance of meta-class Package must be created. Then the name of it must be defined. After the creation of this package one must group the instance of all classes that will be moved to the new package. The refactoring at this point can be considered complete for the package code of KDM. However, now it is necessary to propagate all the changes to other KDM's packages. In Figure~\ref{fig:atl2} is presented a chunk of code responsible for performing the propagation of changes. As can be seen, it is necessary to create an instance of the meta-class Layer, set its name to ``Controller'', see lines X-Y of Figure~\ref{fig:atl2}. Then, the association ``implementation'' of ``Layer'' earlier created need to be specified. In this point it is important to visualise that this association refers to the extracted package, i.e., the package created earlier. Further, it is necessary to create an instance of meta-class AggregatedRelationship. This meta-class has a meta-attribute, ``density'' and a set of association that need to be propagated. The part of the source code responsible to create an AggregatedRelationship is presented in lines 17-23 of Figure~\ref{fig:atl2}. The helper illustrated in lines X-Y represents that the meta-attribute ``density'' is updated always that a new relationship is identified.




We can divide the propagation in three the possible scenarios. The first scenario is presented in Figure X. As noted 



Suppose the system presented in Figure~\ref{fig:system}. This system was developed using the architectural style Model-View-Controller.


