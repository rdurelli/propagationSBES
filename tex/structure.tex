

\begin{abstract}
\linespread{0.87}
%!TEX root = /Users/rafaeldurelli/Dropbox/Artigos Elaborados/KDM propagation_2015/sbes_2015_kdm_propagation/sbes2015_kdm_propagation.tex
%Architecture-Driven Modernization (ADM) is a model-driven alternative to conventional reengineering processes that relies on the Knowledge-Discovery Meta-model (KDM) as the base for the whole process. Unlike conventional meta-models, KDM is capable of putting together different system views (Code, Architecture, Business Rules, Data, Events) in an unique site and also retaining the dependencies among them. During the system life cycle, artifacts tend to change, usually these changes entail refactorings. However, as a system can be represented by several different models, a common accident that arises during refactorings is to desynchronize (inconsistent views) the models. One solution is to apply Static Change Propagation (SCP) techniques. Most of existing SCP techniques deal with propagating changes in different and external models, usually from another vendor preventing or making difficult their application in other models, like KDM. Currently there is a lack of research concentrated on investigating SCP in KDM. In this paper we present a plug-in supported KDM-specific approach for updating dependent models/views when specific elements are refactored. Our approach involve three main steps: \textit{i}) identifying diff between the refactored KDM instance and the original KDM instance (the instance before one applies a KDM refactoring), \textit{ii}) the identification of all affected KDM model elements (dependent on the refactored ones), and \textit{iii}) the propagation of changes in order to keep all the models/views synchronized. We have conducted two evaluation that shows our approach reached good accuracy and completeness levels.
%
Architecture-Driven Modernization (ADM) is a model-driven alternative to conventional reengineering processes that relies on the Knowledge-Discovery Meta-model (KDM) as the base for the whole process. Unlike conventional meta-models, KDM is capable of putting together different views (Code, Architecture, Business Rules, etc) in an unique site and also retaining the dependencies among them. During model-driven development, a system is usually modeled by using several different models, each one representing a particular abstraction view. The general goal is to maintain these views synchronized with each other. A common accident that arises when refactoring models is to desynchronize the views and one solution is to apply Change Propagation (CP) techniques. Most of these techniques are focused on propagating changes in models that conform to different meta-models or from different vendors. Currently there is a lack of research concentrated on investigating CP in KDM. In this paper we present an approach for updating  particular abstraction views when specific KDM elements are refactored. Our approach has three steps: i) identifying diff between the refactored KDM instance and the original one, ii) the identification of all KDM model elements dependent on the refactored ones, and iii) the propagation of changes to keep all the models/views synchronized. We have conducted two evaluation that show our approach reached good accuracy and completeness when propagating the changes.
\end{abstract}
% IEEEtran.cls defaults to using nonbold math in the Abstract.
% This preserves the distinction between vectors and scalars. However,
% if the conference you are submitting to favors bold math in the abstract,
% then you can use LaTeX's standard command \boldmath at the very start
% of the abstract to achieve this. Many IEEE journals/conferences frown on
% math in the abstract anyway.

% no keywords




% For peer review papers, you can put extra information on the cover
% page as needed:
% \ifCLASSOPTIONpeerreview
% \begin{center} \bfseries EDICS Category: 3-BBND \end{center}
% \fi
%
% For peerreview papers, this IEEEtran command inserts a page break and
% creates the second title. It will be ignored for other modes.
\IEEEpeerreviewmaketitle
\section{Introduction}
	\linespread{0.87}
	%!TEX root = /Users/rafaeldurelli/Dropbox/Artigos Elaborados/KDM propagation_2015/sbes_2015_kdm_propagation/sbes2015_kdm_propagation.tex

%In 2003 the Object Management Group (OMG) created a task force called Architecture Driven Modernization Task Force (ADMTF). It aims to analyze and evolve typical reengineering processes, formalizing them and making them to be supported by models [2]. ADM advocates the conduction of reengineering processes following the principles of Model-Driven Architecture (MDA) [22][2], i.e., all software artifacts considered along with the process are models.                                       	

%According to OMG the most important artifact provided by ADM is the Knowledge Discovery Metamodel (KDM). By means of it, it is possible to represent different system abstraction levels by using its models, such as source code (Source and Code models), Actions (Action model), Architecture (Structure Model) and Business Rules (Conceptual Model). The idea behind KDM is that the community starts to create parsers and tools that work exclusively over KDM instances; thus, every tool that takes KDM as input can be considered platform and language-independent, propitiating interchange among tools. For instance, a refactoring catalogue for KDM can be used for refactoring systems implemented in different languages. 

%Central to modernization processes are the refactorings. Refactorings are .....  However, most of existing model-based refactorings do not cope with propagation of the refactoring changes across other dependent abstraction levels, keeping all models synchronized [ , , , , ]

%	In this paper we present Propagation-Aware Refactorings (PARef), an approach for updating dependent models when specific elements are refactored.

In 2003 the Object Management Group (OMG) created a task force called Architecture Driven Modernization Task Force (ADMTF). The goal was to analyze and evolve typical reengineering processes, formalizing them and making them to be supported by models~\cite{1686216}. The result of this effort was the creation of Architecture-Driven Modernization (ADM), which advocates the conduction of reengineering processes following the principles of Model Driven Architecture (MDA)~\cite{Heckel2008, Andrade:2005, Reus:2006}, i.e., all software artifacts considered along with the process are models. Therefore, a typical ADM-based modernization process starts with a reverse engineering phase to recuperate a model representation of the system; proceeds by applying refactorings over the recuperated model and finalize by a forward engineering phase where the modernized system is generated.

Knowledge Discovery Metamodel (KDM) is the most important metamodel provided by ADM. Its main characteristics are: i) it is an ISO-IEC standard since 2009 (ISO/IEC 19506); ii) it is platform/language independent, and ii) it is able to represent different views of the same system and retain the dependencies among them by using specific metaclasses. This third point is possible thanks to several internal KDM metamodels/packages that are focused on specific views or abstraction levels, such as  source-code (Code metamodel), behaviors (Action metamodel), architecture (Structure metamodel), business rules (Conceptual metamodel), database (Data metamodel), events (Event metamodel), Graphical User Interface (GUI) (UI metamodel) and deployment (platform metamodel).  

It is well known that refactoring activities are central to modernization processes. Refactorings are defined as the process of modifying the internal structure of software without changing its external observable behavior~\cite{refactImpro}. Behavior preservation in refactoring activities has received a lot of attention for years, both in source code and in models~\cite{4440135, Mens:2006:TMT:1706639.1706924, Mens:2006_NEW, Mens:2007}. One of the known problems when refactoring models is change propagation, i.e., the modifications that need to be done in model elements that are dependent on the refactored model element. Although the behavior preservation is harder to check and characterize when dealing with models, there are works that present proposals of keeping the behavior models updated when static models are refactored~\cite{ICSOFT2014_Winetzhammer}. Most of the works propose solutions to propagate changes across different metamodels not in the same metamodel. 

%However, although some research has been conducted on the theme of change propagation in models~\cite{4440135, Mens:2006:TMT:1706639.1706924, Mens:2006_NEW, Mens:2007, ICSOFT2014_Winetzhammer}, none of them have devoted attention on a metamodel like KDM, which groups several metamodels under a unique place and already provide metaclasses for retaining the dependences among these models. In most cases, the related works concentrate on propagating changes in a metamodel different from where had occurred the modification. Besides, the concentration of some of them are in behavior preservation, an aspect that is out of the scope of this work.

However, although some research has been conducted on the theme of change propagation in models~\cite{4440135, Mens:2006:TMT:1706639.1706924, Mens:2006_NEW, Mens:2007, ICSOFT2014_Winetzhammer}, none of them has devoted attention on a metamodel like KDM, which groups several metamodels under a unique place and already provide metaclasses for retaining the dependences among these models. In most cases, the related works concentrate on propagating changes in a different metamodel from where had occured the modification. Besides, the concentration of some of them are in behavior preservation, an aspect that is out of the scope of this work. Furthermore, up to this moment, few research has been done on KDM refactorings~\cite{IRIDurelliCatalogo}, limiting the dissemination and adoption of ADM. We believe that our change propagation approach will foster the creation and research on KDM refactorings. 

%In this paper we present an approach for propagating changes when refactoring KDM model instances. The main goal is to guarantee the global system representation keep synchronized along with the refactoring activities; which are much common during modernization processes. Our approach runs in three steps: i) a search algorithm identifies all the KDM metaclasses dependent on the refactored elements and ii) an ATL Transformation Language (ATL) that performs the model transformation/refactoring, and iii) another ATL created to propagate all the changes throughout all KDM's view. This last step was implemented in a generic way - as a decoupled module, which can be coupled to existing refactorings. In this way, existing users can write KDM refactorings in ATL without worrying about the change propagation. The only task is to provide for our component the input it needs to conduct the propagation.

In this paper we present an approach for propagating changes when refactoring KDM model instances. The main goal is to guarantee the global system representation keep synchronized along with the refactoring activities; which are much common during modernization processes. Our approach runs in three steps: i) a mining algorithm identifies all KDM metaclasses that need to be updated when refactoring a specific KDM metaclass, ii) an ATL Transformation Language (ATL) that performs the intended refactoring, and iii) another ATL that performs one or more model transformations that characterize change propagation. We have implemented the approach in a generic way as a decoupled module, which can be coupled to existing refactorings. In this way, existing users can write KDM refactorings in ATL without worrying about the change propagation. The only task is to provide for our component the input it needs to conduct the propagation.

The main contributions are: i) a mining algorithm to identify all KDM metaclasses that need to be updated when a specific refactoring is performed, ii) a set of refactoring devised to KDM domain, iii) a propagation technique approach, and (iv) a support and preliminary infrastructure for allowing the creation of refactorings for kdm.

This paper is structured as follows: In Section~\ref{sec:background} the notion related to ADM and KDM, their details and a system's description that was instantiated in KDM are showed.
%
%
%Section~\ref{sec:background} presents all needed background on ADM and KDM. 
In Section~\ref{sec:motivation_and_running_example} a motivation is presented. Section~\ref{sec:the_approach} shown the proposed approach. In Section~\ref{sec:evaluation}, an empirical evaluation is presented. In Section~\ref{sec:related_work} there are related works and in Section~\ref{sec:conclusion} there are the conclusions.



	%This paper is structured as follows: in Section II, Crosscutting Frameworks are explained; in Section III, the Proposed Model and the Reuse Process are shown; in Section IV, a tool to support the process is used to reuse a persistence framework as an Example; in Section V, an empirical evaluation is presented; in Section VI, there are related works and in Section VII, there are the conclusions.

\section{ADM and KDM} \label{sec:background}
	\linespread{0.87}
	%!TEX root = /Users/rafaeldurelli/Dropbox/Artigos Elaborados/KDM propagation_2015/sbes_2015_kdm_propagation/sbes2015_kdm_propagation.tex

\section{Background} % (fold)
\label{sec:background}

In this section we provide a brief background to Architecture-Driven Modernization (ADM) and Knowledge Discovery Metamodel (KDM). Further, we describe in detail why change propagation in KDM is a complex process.

\subsection{ADM and KDM}

The growing interest in using Model-Driven Development (MDD) to manage software evolution is mainly focused on the reengineering or modernization of legacy systems. Several software migration projects have been carried out with model-driven approaches~\cite{Heckel2008, Andrade:2005, Reus:2006}. %In addition, the 

This growing interest motived OMG to define the ADM initiative~\cite{1686216} which advocates carrying out the reengineering process considering MDD principles. 
ADM is the concept of modernizing existing systems with a focus on all aspects of the current systems architecture and the ability to transform current architectures to target architectures by using all principles of MDD~\cite[p.~60]{Ulrich:2010:IST:1841736}. 


Figure~\ref{fig:ADM_shorseshoe} depicts the horseshoe model (i.e., horseshoe is basically a left-hand
side, a right-hand side and a bridge between the sides) which was adapted to ADM. %Please note that it contains all the traditional phases and some MDD's keywords, such as PSM and  PIM. The traditional phases adapted to ADM are:

\begin{itemize}

\item \textbf{Reverse Engineering}: herein a reverse engineering is realized, it takes a legacy system to be modernized as input, then the knowledge is extracted and a PSM is generated. In addition the PSM serves as the basis for the generation of a Platform-Independent Language (PIM), which is called KDM;

\item \textbf{Restructuring}: in this phase a set of restructuring/refactoring can be applied into a KDM's instance by means of M2M transformations;

\item \textbf{Forward Engineering}: then a forward engineering is carried out and the source code of the modernized target system is generated.

\end{itemize} 

%Figure~\ref{horseshoe} depicts the ADM modernization domain model where the left side of the horseshoe is the current state of a busines/it architecture ``as is'' and the right side is what we want to get after the modernization ``to-be''. %One common path followed by IT is to focus on the technical architecture. Generally, the cost of this approach is lower and project duration shorter because the data and application architectures remain largely intact but there is almost no impact or value to the business. 

%On the other hand, a modernization which seeks to provide value to the business, would need to change the application and data architecture, which in turn would rely on an analysis of requirements stemming from shifts to the business architecture. These types of projects are of a longer duration, require more investment, and deliver significantly more value to the business.


\begin{figure}[!ht]
\centering
  % Requires \usepackage{graphicx}
 \includegraphics[scale=0.55]{figuras/horseshoes}
\caption{Horseshoe Modernization Model. This figure is adapted from~\cite{OMG_ADM}.}
\label{fig:ADM_shorseshoe}
\end{figure}

In order to perform a systematic modernization as depicted in Figure~\ref{fig:ADM_shorseshoe}, ADM introduces several modernization standards, among them there is the Knowledge Discovery Metamodel (KDM).
%However, herein we focus on KDM because it is the key cornerstone of ADM and the main ideas of our research. 
KDM is an OMG specification adopted as ISO/IEC 19506 by the International Standards Organization for representing information related to existing software systems. 
The goal of the KDM standard is to define a metamodel to represent all the different legacy software artifacts involved in a legacy information system (e.g. source code, user interfaces, databases, business rules, etc.). %The metamodel of the KDM standard provides a comprehensive high-level view of the behavior, structure and data of legacy information systems by means of a set of facts. 

KDM contains twelve packages and it is structured in a hierarchy of four layers: (\textit{i}) Infrastructure Layer, (\textit{ii}) Program Elements Layer, (\textit{iii}) Runtime Resource Layer, and (\textit{iv}) Abstractions Layer. These layers are created automatically, semi-automatically or manually through the application of various techniques of extraction of knowledge, analysis and transformations~\cite{1686216}. Figure~\ref{fig:kdmLayers} depicts the architecture of KDM. By observing this figure it is fairly evident that each layer is based on the previous layer, thus, they are organized into packages that define a set of metamodel, whose purpose is to represent a specific and independent interest of knowledge related to legacy systems, e.g. source code, user interfaces, databases, business rules, etc.

\begin{figure}[!ht]
\centering
  % Requires \usepackage{graphicx}
 \includegraphics[scale=0.46]{figuras/kdm_layer}
\caption{KDM Architecture.}
\label{fig:kdmLayers}
\end{figure}

Although KDM is a metamodel to represent a whole system, its main purpose is not the representation of models related strictly to the source code nature such as Unified Modeling Language (UML). While UML can be used to generate new code in a top-down manner, an ADM-based process using KDM starts from the different legacy software artifacts and builds higher-abstraction level models in a bottom-up manner through reverse engineering techniques. %KDM can be seen from different perspectives, as follows: (\textit{i}) KDM can be considered as a metamodel to represent legacy knowledge models, (\textit{ii}) most of the KDM specification is a definition of a language- and platform-independent ontology of legacy information systems and (\textit{iii}) KDM is a common interchange format that makes the interoperability between the reverse engineering tools and modernization tools possible.

%KDM specification owns some KDM domain, each domain defines an architectural viewpoint. In order to define the catalogue of refactoring for the KDM we need to focus just on the Program Element Layer - more specifically  in the Code Package, which represents the code elements of a program (classes, fields and methods) and their associations. %We are interested in the Code Package once our catalogue is based on fine-grained refactorings, i.e., refactorings to be applied into classes, fields and methods. 
%Therefore, it is important to dig a little deeper in the Code Package.
%
%In a given KDM instance, each instance of the code meta-model element represents some programming language construct, determined by the programming language of the existing software system. Each instance of a code meta-model element corresponds to a certain region of the source code in one of the artifacts of the existing software system. In addition, 
%
%The Code Package consists of $24$ classes and contains all the abstract elements for modeling the static structure of the source code. In Table~\ref{tab:mappingCodeToKDM} is depicted some of them. This table identifies KDM metaclasses possessing similar characteristics to the static structure of the source code. Some metaclasses can be direct mapped, such as Class from object-oriented language, which can be easily mapped to the ClassUnit metaclass from KDM.


%\begin{table}[!h]
%\caption{Metaclasses for modeling the static structure of the source-code}
%\label{tab:mappingCodeToKDM}
%\centering
%\begin{tabular}{|>{\centering}p{3cm}|>{\centering}p{3cm}|}
%\hline 
%Source-Code Element & KDM Element\tabularnewline
%\hline 
%\hline 
%Class & ClassUnit\tabularnewline
%\hline 
%Interface & InterfaceUnit\tabularnewline
%\hline 
%Method & MethodUnit\tabularnewline
%\hline 
%Field & StorableUnit\tabularnewline
%\hline 
%Local Variable & Member\tabularnewline
%\hline 
%Parameter & ParameterUnit\tabularnewline
%\hline 
%Association & KDM RelationShip\tabularnewline
%\hline 
%\end{tabular}
%\end{table}

  %\begin{figure}[!ht]
  %\centering
  % Requires \usepackage{graphicx}
    %\includegraphics[scale=0.39]{FIGURAS_DA_REFATORACAO/ProgramLaye0r}
  %\caption{Chunk of the Code Package (OMG Group~\cite{OMGADM})}
  %\label{fig:programLayer}
  %\end{figure}

 %As can be seen in Figure~\ref{fig:programLayer} the root metaclass is \textit{ComputationalObject} which has two sub-metaclasses, i.e., \textit{DataElement} and \textit{ControlElement}. The former sub-metaclass, \textit{DataElement}, is a generic modeling element that defines the common properties of several concrete classes that represent the named data items of existing software systems, for example, global and local variables, record files, and formal parameters. \textit{DataElement} has five sub-metaclasses - \textit{StorableUnit}, \textit{IndexUnit}, \textit{ItemUnit}, \textit{ParameterUnit} and \textit{MemberUnit}. \textit{StorableUnit} is a concrete  sub-metaclass of the \textit{StorableElement} meta-class that represents variables of the existing software system. \textit{IndexUnit} class is a concrete subclass of the \textit{DataElement} class that represents an index of an array datatype. Instances of \textit{ItemUnit} class are endpoints of KDM data relations which describes access to complex datatypes. \textit{ParameterUnit} class is a concrete subclass of the \textit{DataElement} class that represents a formal parameter; for example, a formal parameter of a procedure. \textit{MemberUnit} class is a concrete subclass of the \textit{DataElement} class that represents a member of a class type. Finally, the latter, \textit{ControlElement} is a sub-metaclass that contains two sub-metaclasses - \textit{MethodUnit} and \textit{CallableUnit}. \textit{MethodUnit} element represents member functions owned by a \textit{ClassUnit}, including user-defined operators, constructors and destructors. The \textit{CallableUnit} represents a basic stand-alone element that can be called, such as a procedure or a function. %As can be seen below the dashed line in Figure~\ref{fig:programLayer} there are also the following enumerations: ``\textit{ExportKind}'', ``\textit{StorableKind}'', ``\textit{CallableKind}'', ``\textit{MethodKind}'', which are sets os literals used as properties of the metaclasses.
 

In order to show how KDM and its metaclasses can be used to represent a system, please considerer a toy system, which is depicted in Figure~\ref{fig:system}. Also, note that this system is used throughout this paper as a running example. 

\begin{figure*}
	\centering
	% Requires \usepackage{graphicx}
	\includegraphics[scale=0.58]{figuras/NewSystemVersion}
	\caption{Motivation and running example.}
	\label{fig:system}
\end{figure*}

This toy system is based on a well know Model View Controller (MVC) design pattern. As noted in Figure~\ref{fig:system} it is split in four KDM levels/packages, which are illustrated in the figure bounded by dashed lines shape. Following is described each KDM levels/packages and its meaning regarding to the illustrated system.

\begin{itemize}

\item Code Package - represents the source-code (physical artifacts). In Figure~\ref{fig:system} is is possible to see three packages: (\textit{i}) \texttt{GUI}, (\textit{ii}) \texttt{CTR}, and (\textit{iii} \texttt{Model}). The first one, \texttt{GUI}, contains four classes, \texttt{StudentGUI}, \texttt{InstructorGUI}, \texttt{Student}, and \texttt{Instructor}. The second package contains three classes: \texttt{StudentCTR}, \texttt{UtilCon}, \texttt{InstructorCTR}. Then, the third one owns two classes, \texttt{Secretary} and \texttt{Researcher}. Note that these classes are related to each other by means of primitive relationships, such as: \texttt{Calls}, \texttt{Creates}, \texttt{Extends}, etc;

\item Structure Package - illustrates the system's architecture, herein the system is based on MVC. As noted in Figure~\ref{fig:system} each rectangle depicts a layer, i.e., \texttt{View}, \texttt{Controller}, and \texttt{Model}. For the first layer, we have associated \texttt{View} with the package \texttt{GUI}. In KDM this kind of association is done by means of the meta-attribute \texttt{implementation}, which are depicted by the dashed arrows. Similarly, \texttt{Controller} was associated with the package \texttt{CTR}, and the layer \texttt{Model} was associated with the package \texttt{Model}, respectively. Regarding to the relationships among the layers, it is possible to visualize pipes between two layers (see Figure~\ref{fig:system}. These pipes represents the corresponding aggregated relationship, which represents the number summing all primitive relationships among layers. For instance, the aggregation relationship between the layer \texttt{View} and the layer \texttt{Controller} are represented by the relationships: \texttt{Calls}, \texttt{Creates}, \texttt{Extends}, and another \texttt{Calls} from the Code Package. Summing up these relationships the density value is 4. Following the same idea the relationship between the layer \texttt{Controller} and layer \texttt{Model} is 2;  
  
\item Conceptual Package - illustrates the system's business rules domain. Note that this system owns three scenarios, each of them are associated with a package from Code Package by means of the association \texttt{implementation}, see the dashed arrows. Further, each scenario contains a rule except the last one. In it turn, each rule is associated with a class from Code package, again using the association \texttt{implementation};

\item Data Package - depicts the system's database and its tables. Herein, it is possible to notice that the depicted system owns a set of Plain Old Java Objects (POJOS), they are: Student, Instructor, Secretary, and Researcher. All of these POJOS are also Object Relational Mapping (ORM), i.e., they are mapped to the Data package using the metaclasse RelationalTable. 

\end{itemize}

%(\textit{i}) Code Package, which represents the source-code (physical artefacts), (\textit{ii}) Structure Package, that illustrates the system's architecture, herein the system is based on MVC, (\textit{iii}) Data Package, which depicts the system's database and its tables, and (\textit{iv}) Conceptual Package, which is intended to be the basis for formal and detailed natural language declarative description of a complex entity, such as a business


%owns three layers, they are: (\textit{i}) ``View'', (\textit{ii}) ``Controller'', and (\textit{iii}) ``Model''. Inside of each layer there is at least one package. For instance, the relationships ``GUI inside View'', ``CTR inside Controller'', and ``model inside Model'' mean that ``GUI'', ``CTR'' and ``model'' are contained in ``View'', ``Controller'' and ``Model'', respectively or in some sub-container of  them, transitively. Similarly, the relationship ``StudentGUI inside GUI'' means that ``StudentGUI'' is in container ``GUI'' or in some sub-container of ``GUI''.



%Furthermore, it is possible to notice that the depicted system owns a set of Plain Old Java Objects (POJOS), they are: ``Student'', ``Instructor'', ``Secretary'', and ``Researcher''. All of these POJOS are also Object Relational Mapping (ORM).

Considering this system it is possible to highlight some problem or even to add new requirements. For instance, a problem that can be noticed is that both classes \texttt{Student} and \texttt{Instructor} should be contained in \texttt{Model} package not in \texttt{GUI} package, respectively. In order to fix this problem, one should apply a refactoring - for instance, \textit{Move Class}.
%

%

Regarding to a new requirement let's pretend someone has identified that the class \texttt{Student} is doing work that should be done by two classes - it contains attributes to hold informations upon student's addresses. In order to fulfill this new requirement one should apply the refactoring \textit{Extract Class} and creates a new class named Address (which is a POJO and also an ORM) and move all student's attributes related to address to this new class.

%The action of this refactoring should propagate throughout other KDM's levels, such as the data level\footnote{The KDM's level that contains information on data base schema, table, column, primary key, etc}. 

However, in both described refactoring it is necessary a skilled domain expert into KDM to identify all the metaclasses in the system which involve/reference the classes aforementioned and correct them respectively in all KDM packages, i.e., propagate all refactoring's impact throughout all KDM's packages. 

For instance, considering the refactoring \textit{Move Class} (move \texttt{Student} and \texttt{Instructor} from \texttt{GUI} package to \texttt{Model} package) changes should be propagated to the Structure Package and to the Conceptual Package to maintain the model synchronized. For instance, the \texttt{density}, i.e., aggregation relation ship between the layer \texttt{View} and the layer \texttt{Controller} would change from 4 to 2 - once the primitives relationships \texttt{Create} and \texttt{Extends} would no longer exist from the package \texttt{GUI} to the package \texttt{CTR}. On the other hand, the resulting of this refactoring would update the density between the layer \texttt{Model} and \texttt{Controller}, instead of 2 it should be 4, as \texttt{Creates} and \texttt{Extends} were also moved along with its classes, \texttt{Student} and \texttt{Instructor}. Concerning to the Conceptual package, the  RuleUnit\_1.1 that is associated with \texttt{Student} should also be moved to ScenarioUnit\_3. 

Regarding to the refactoring \textit{Extract Class}, the extracted class \texttt{Address} would be a POJO (it would be contained in Model package) and it would also be an ORM - therefore, the action of this refactoring should be propagated throughout  the Data package, i.e., the instance of \texttt{Address} should be associated with a metaclass \texttt{RelationalTable}, and its attributes should be associated with  of \texttt{ColumnSet}.

%, i.e., the relationship among the layers should be propagated automatically. Similarly, considering the refactoring \textit{Extract Class}, where a new POJO and ORM class is created, the data's level also should be propagated.

These propagation seen to be easy to apply, however, in a complex system comprising all kdm's packages/levels, propagate all changes after a refactoring is a difficult and error-prone task. Even identifying the affected parts of the KDM's packages/levels is not an easy and straightforward process. In order to fulfill this limitation and create an automatized process we have devised Propagation-Aware Refactorings (PARef) that contains three main steps. The first step is the identification of all dependent elements related to a specific refactoring. In the second step the refactoring of all identified elements are performed using a model-to-model transformation language - the third step is the propagation of changes in order to keep all the dependent models synchronized. %In the following sections we show in detail that change propagation in KDM is a complex process that can be solved semi-automatically and, hence, efficiently and precisely if we provide a rigorous theoretical background. 
In the following sections we show the theoretical background need to fully understand our approach. Then, our approach is presented in Section~\ref{sec:the_approach}.


%  one should identify and change all the relationship after the refactoring

%Let us consider a new user requirement that the both classes ``Student'' and ``Instructor'' should not longer be represented in the layer ``View''. Instead, these classes should be allocated into a new layer, e.g., layer ``Model''. This new requirement would require the application of a refactoring - for instance, \emph{Extract Package}. However, such a situation would require a skilled domain expert into KDM to identify all the metaclasses in the system which involve/reference the classes aforementioned and correct them respectively in all KDM levels. Apparently, in a complex system comprising all kdm's levels, this is a difficult and error-prone task. Even identifying the affected parts of the KDM's levels is not an easy and straightforward process. 

%Our approach contains three main steps. The first one is the identification of all dependent elements. In the second step the refactoring of all identified elements are performed and the third one is the propagation of changes in order to keep all the dependent models synchronised. In the following sections we show in detail that change propagation in KDM is a complex process that can be solved semi-automatically and, hence, efficiently and precisely if we provide a rigorous theoretical background. 
 


\begin{figure}
	\centering
	% Requires \usepackage{graphicx}
	\includegraphics[scale=0.53]{figuras/TreeNewJoint}
	\caption{A bird's eye view of a KDM's instance.}
	\label{fig:allKDMLayers}
\end{figure}

\subsection{Change Propagation in KDM} % (fold)
\label{sub:change_propagation_in_kdm}

In our previous work~\cite{IRIDurelliCatalogo}, we introduced a refactoring catalogue for KDM for managing evolution of a software system. This paper served as a starting point to investigate how the changes affect the KDM's levels. For instance, depending on the refactoring a set of metaclasses must to be create, updated and even removed, these operations may cause minor or major changes to be propagated into other KDM's metaclasses. In order to explain the propagation of changes in KDM consider the Figure~\ref{fig:allKDMLayers}. This figure depicts the corresponding, though simplified KDM instance of system depicts in Figure~\ref{fig:system}. It illustrates a KDM instance as a UML object diagram for the sake of simplicity. Notice that due space limitations some elements are not depicted in this figure.

As we can see, a KDM's instance can be understood as a tree where we have a specially node called the root of the tree. Then the remaining nodes are partitioned into $\textit{M} >= 0$ joint sets $T_{1}, T_{2}, ..., T_{n}$, and each of these sets is a subtree.  Each nodes represent a metaclass that make up the system depicted in Figure~\ref{fig:system}. The edges represent the relationship between the metaclasses.


%each KDM's levels/packages can be partitioned both horizontally and vertically; in both cases its metaclasses are closely related and interconnected. %The relations form the key concept of modernization by means of KDM, since they invoke the needs for change propagation. 
%
The root is the metaclass \texttt{Segment}. There are four subtrees rooted at \texttt{StructureModel}, \texttt{CodeModel}, \texttt{ConceptualModel}, and \texttt{DataModel}, respectively. 
The tree rooted at \texttt{StructureModel} has three \texttt{Layers}, \texttt{CONTROLLER}, \texttt{VIEW}, and \texttt{MODEL} - they are connected by the metaclasses \texttt{AggregatedRelationship} (see Figure~\ref{fig:system} and Figure~\ref{fig:allKDMLayers}).

The tree rooted at \texttt{CodeModel} has three instance of the metaclass \texttt{Package} - \texttt{CRT}, \texttt{GUI}, and \texttt{MODEL}, respectively. Further, each package contains a set of classes, for instance, the package \texttt{MODEL} has two instance of the metaclass \texttt{ClassUnit}, \texttt{Researcher}, and \texttt{Secretary}, respectively.

The tree rooted at \texttt{ConceptualModel} also has three subtree - herein represented by the metaclass \texttt{ScenarioUnit}. Further, each node of a tree is the root of a \texttt{RuleUnit}. Finally, the \texttt{DataModel} has one subtree - \texttt{RelationalSchema}, which represent the system's data base schema. It contains four subtree - \texttt{Secretary}, \texttt{Researcher}, \texttt{Instructor}, and \texttt{Student}, where each node is an instance of the metaclass \texttt{RelationalTable}. 


%If we consider the horizontal partitioning, we can identify four branches. Each branches represents a KDM's levels/packages just as in the Figure~\ref{fig:system}. The lower branch, \texttt{StructureModel}

% However, in this Figure, each KDM's levels/packages are grouped by its root elements. For instance, the node labeled \texttt{StructureModel} is the root metaclass for all architecture elements.  


%on the lower level it is possible to see the metaclasse \texttt{StrutureModel} that is the root metaclasses of all architecture elements. As can be seen it contains three Layers where each bounded dashed lines shape represents a KDM's levels/package.



% . Each package is used to represent an specific artifact, such as, source-code, structure elements, databases, and conceptual elements. As stated in Section~\ref{sec:background} KDM is organized into four layers. In Figure~\ref{fig:allKDMLayers} three of them is shown. The squiggle (leftmost) part represents the Abstract Layer, which defines a set of meta-model elements that represent domain-specific and application-specific abstractions. The stipple (middle) part represents the Resource Layer, which describes common patterns for representing the operating environment of existing software systems. Finally, the highlighted in grey (rightmost) defines a large set of meta-model elements whose purpose is to provide a language-independent intermediate representation for various constructs determined by common programming languages.
 

%Further, if we consider the horizontal partitioning, we can identify four levels, each representing a different view of an system instantiated in KDM. The lowest level represents all the physical artifacts owned in a system. Its parent level, called KDM's metaclasses, represents the particular KDM's metaclasses, which conforms to an specific artifact. The level above represents the system specification. Finally, the highest level represent all the KDM's models, each model defines a set of metaclasses to represent the physical and logical elements of software as well as their relationships at various levels of abstraction.

%In order to cope with propagation of the refactoring changes across any KDM's subtree dependent abstraction levels, keeping all models synchronized

In the context of model-driven refactoring, if any change occurs at any KDM's subtree the change should be propagated to other elements.
%
%These levels indicate problems related to KDM propagation of changes. 
%
For instance, when the elements of \texttt{CodeModel} suffer any kind of changes (e.g., are refactored), its instances, i.e., \texttt{ClassUnits}, \texttt{MethodUnits}, \texttt{StorableUnits}, etc, and related elements must be adapted accordingly so that their validity and correctness is preserved respectively. In addition, if we want to preserve others parts of KDM, like the system's structure and the business rules the  \texttt{StructureModel} and \texttt{ConceptualModel} also need to adapt, respectively. %What is more, as we have mentioned, in practice there are usually at least one instance of each KDM's model applied in a single system, e.g., the system architecture conforms to Structure Model, the source-code conforms to Code Model, etc. 
In general, a change at one KDM's model should trigger a cascade of changes at other models. We call such sequences of adaptations change propagation.

As we can see in Figure~\ref{fig:allKDMLayers}, there are not only horizontally relations between the models, but the elements of the system can also be vertical related across the vertical partitions. A few examples are denoted by the red/blue dashed arrows. For instance, there is a relation between a \texttt{CodeModel} (its respective metaclasses) with the \texttt{StructureModel} - which means that a change in one of the ends of the relation can influences the other.

Considering these KDM's models leads to evolution of each affected model separately. However, this is a highly time- consuming and error-prone solution since we need a domain expert who is able to identify all the affected models and propagate the changes. Following we present our approach to detect and propagate all the changes throughout all KDM's levels.

	

%%!TEX root = /Users/rafaeldurelli/Dropbox/Artigos Elaborados/KDM propagation_2015/sbes_2015_kdm_propagation/sbes2015_kdm_propagation.tex
\section{Change Propagation in KDM} % (fold)
\label{sec:motivation_and_running_example}

In our previous work~\cite{IRIDurelliCatalogo}, we introduced a refactoring catalogue for KDM for managing evolution of a software system. This paper served as a starting point to investigate how the changes affect the KDM's levels. 

Considering the system described earlier it is possible to highlight some problem or even to add new requirements that will propagate changes at KDM's levels. For instance, a problem that can be noticed is that both classes \texttt{Student} and \texttt{Instructor} should be contained in \texttt{Model} package not in \texttt{GUI} package, respectively. In order to fix this problem, one should apply the refactoring \textit{Move Class}.
%
%
%
%
Regarding to a new requirement let's pretend someone has identified that the class \texttt{Student} is doing work that should be done by two classes - it contains attributes to hold informations upon student's addresses. In order to fulfill this new requirement one should apply the refactoring \textit{Extract Class} and creates a new class named Address (which is a POJO and also an ORM) and move all student's attributes related to address to this new class.

%The action of this refactoring should propagate throughout other KDM's levels, such as the data level\footnote{The KDM's level that contains information on data base schema, table, column, primary key, etc}. 

In both described refactoring it is necessary a skilled domain expert into KDM to identify all the metaclasses in the system which involve/reference the classes aforementioned and correct them respectively in all KDM packages, i.e., propagate all refactoring's impact throughout all KDM's packages. 

For instance, considering the refactoring \textit{Move Class} (move \texttt{Student} and \texttt{Instructor} from \texttt{GUI} package to \texttt{Model} package) changes should be propagated to the Structure Package and to the Conceptual Package to maintain the model synchronized. For instance, the \texttt{density}, i.e., aggregation relation ship between the layer \texttt{View} and the layer \texttt{Controller} would change from 4 to 2 - once the primitives relationships \texttt{Create} and \texttt{Extends} would no longer exist from the package \texttt{GUI} to the package \texttt{CTR}. On the other hand, the resulting of this refactoring would update the density between the layer \texttt{Model} and \texttt{Controller}, instead of 2 it should be 4, as \texttt{Creates} and \texttt{Extends} were also moved along with its classes, \texttt{Student} and \texttt{Instructor}. Concerning to the Conceptual package, the  RuleUnit\_1.1 that is associated with \texttt{Student} should also be moved to ScenarioUnit\_3. 

Regarding to the refactoring \textit{Extract Class}, the extracted class \texttt{Address} would be a POJO (it would be contained in Model package) and it would also be an ORM - therefore, the action of this refactoring should be propagated throughout  the Data package, i.e., the instance of \texttt{Address} should be associated with a metaclass \texttt{RelationalTable}, and its attributes should be associated with  of \texttt{ColumnSet}.

%, i.e., the relationship among the layers should be propagated automatically. Similarly, considering the refactoring \textit{Extract Class}, where a new POJO and ORM class is created, the data's level also should be propagated.

These propagation seen to be easy to apply, however, in a complex system comprising all kdm's packages/levels, propagate all changes after a refactoring is a difficult and error-prone task. Even identifying the affected parts of the KDM's packages/levels is not an easy and straightforward process. 

%-------------------------
In the context of model-driven  refactoring, if any change occurs at any KDM's subtree the change should be propagated to other elements.
%
%These levels indicate problems related to KDM propagation of changes. 
%
For instance, when the elements of \texttt{CodeModel} suffer any kind of changes (e.g., are refactored), its instances, i.e., \texttt{ClassUnits}, \texttt{MethodUnits}, \texttt{StorableUnits}, etc, and related elements must be adapted accordingly so that their validity and correctness is preserved respectively. In addition, if we want to preserve others parts of KDM, like the system's structure and the business rules the  \texttt{StructureModel} and \texttt{ConceptualModel} also need to adapt, respectively. %What is more, as we have mentioned, in practice there are usually at least one instance of each KDM's model applied in a single system, e.g., the system architecture conforms to Structure Model, the source-code conforms to Code Model, etc. 
In general, a change at one KDM's model should trigger a cascade of changes at other models. We call such sequences of adaptations change propagation.

As we can see in Figure~\ref{fig:allKDMLayers}, there are not only horizontally relations between the models, but the elements of the system can also be vertical related across the vertical partitions. A few examples are denoted by the red/blue dashed arrows. For instance, there is a relation between a \texttt{CodeModel} (its respective metaclasses) with the \texttt{StructureModel} - which means that a change in one of the ends of the relation can influences the other.

Considering these KDM's models leads to evolution of each affected model separately. However, this is a highly time- consuming and error-prone solution since we need a domain expert who is able to identify all the affected models and propagate the changes. Following we present our approach to detect and propagate all the changes throughout all KDM's levels.



%-------------------------

In order to fulfill this limitation and create an automatized process we have devised Propagation-Aware Refactorings (PARef) that contains three main steps. The first step is the identification of all dependent elements related to a specific refactoring. In the second step the refactoring of all identified elements are performed using a model-to-model transformation language - the third step is the propagation of changes in order to keep all the dependent models synchronized. %In the following sections we show in detail that change propagation in KDM is a complex process that can be solved semi-automatically and, hence, efficiently and precisely if we provide a rigorous theoretical background. 
In the following sections our approach is presented.
 
	
\section{The Propagation Approach} \label{sec:the_approach}
	\linespread{0.87}
	%!TEX root = /Users/rafaeldurelli/Dropbox/Artigos Elaborados/KDM propagation_2015/sbes_2015_kdm_propagation/sbes2015_kdm_propagation.tex
%
\section{Propagation-Aware Refactorings} % (fold)
\label{sec:the_approach}

\begin{figure*}[t]
	\centering
	% Requires \usepackage{graphicx}
	\includegraphics[scale=0.545]{figuras/StepsApproach}
	\caption{Propagation-Aware Refactorings steps.}
	\label{fig:approach}
\end{figure*}

In this section our approach named Propagation-Aware Refactoring (PARef) is presented. PARef aims to propagate changes, in a cascade way, throughout all the KDM's levels during a refactoring. The intention is to keep the consistency/synchronization among all the KDM's views during a refactoring activities.

% in order to keep all the KDM views synchronized. The intention is to keep the consistency/synchronization among all the KDM's views during a refactoring activities. 

%In order to fulfill the limitation pointed out, we introduce a tool/approach, called Propagation-Aware Refactoring (PARef),  that aims propagating changes throughout all the KDM's levels during a refactoring in order to keep all the views synchronized. The intention is to keep the consistency among all the KDM's views during refactoring activities.
%
Figure~\ref{fig:approach} shows the workflow of our approach. As noted it contains three steps, [A], [B], and [C] all contained into the gray cube. Outside of our approach there is the Refactoring activity, see the white rectangle. This activity is a normal and conventional model refactoring activity - this activity is out of scope of our approach is responsibility of the software modernizer to develop or reuse any model refactorings (in ATL, ETL, QVT) and apply them into a KDM model. %In this phase, a number of KDM model elements can be modified, created or removed. 
After that, the first step, [A] is trigged then a diff between the refactored KDM instance and the original KDM instance (the instance before one applies a KDM refactoring). The output of this diff is a list that contains all KDM model elements involved during the KDM refactoring. %Further our two-steps approach starts.  
From this point onward, the step [B], called \textit{Identifying Points to Propagate} aims to gather all the KDM elements that need to be updated/synchronized as a result of a refactoring. %This step uses the list that was obtained from the diff between the refactored KDM and original KDM instance (step [A]) as input. 
This step runs a depth-first search algorithm\footnote{From here on in Dependents Identification Algorithm (DI Algorithm)}. This algorithm uses as input the list that was obtained from the diff between the refactored KDM and original KDM instance (step [A]).
Then, It also uses the refactored KDM instance to generate as output all meta-classes that possesses dependencies with the elements to be refactored. The third step [C], called \textit{Performing Propagation}, objectives to effectively perform in a cascade way all the changes/updates in the KDM model. The input for this steps are the elements to be changed (provided by the step [B]) and the output is the KDM model updated/synchronized. 

The step [A], is technically supported by the framework EMF Compare once it provides comparison and merge facility for any kind of model. EMF Compare was reused and extended to compare instances of KDM models. The step [B], is technically supported by an \textit{Identification Engine} whose the core is a depth-first search algorithm along with a set of queries that are performed over a KDM model. %The \textit{Identification of the Points To Be Propagated} is easier to be done after the refactoring and the after performing the diff between , because it is possible to detect the already changed elements making a diff between the source and refactored model. %When the identification process should be done before the refactoring application, all the model elements involved in the refactoring need to be extracted and passed as parameters to the Identification Engine. This will require more implementation effort.

The step [C] is technically supported by a \textit{Propagation Engine}, whose core is a set of pre-defined transformation rules devised with ATL that works in cascade way. Herein all the transformation rules act as a chain of transformations that are executed together in order to update/propagate all the changes throughout KDM's views. More details on each step are provided in the next sections.


%This approach ensures that when a change/refactoring is performed in any KDM's level, it is correctly propagated to the affected KDM's levels and vice versa. So it make certain that the consistency between the KDM's levels when they are refactored. Our approach is called Propagation-Aware Refactoring (PARef) and it is split into three steps, which are depicted by its corresponding letters and tittle in Figure~\ref{fig:approach}. 

%According to the literature, there are two possibilities to perform propagation in models (colocar ref). One possibility is to mark the modifications on the source KDM's model without actually commiting them until the end of the transformation, so that expression evaluation can occur on the original source KDM's model by ignoring the modifications. Another possibility is to first compute the set of basic operations to perform, storing this set in an external artifact representation and then apply all the changes at once. Our approach follows the second possibility and it is divided in three steps, which are depicted by its corresponding letters and tittle in Figure~\ref{fig:approach}.


%In step [A], \textit{Apply Refactoring}, here the software modernizer has to choose an appropriate refactoring to be applied into the KDM. In this step, new metaclasses can be created, updated, and removed. Also it is necessary to gather all the needed parameters for applying the refactoring. %that the software modernizer inputs all the needs parameters for applying the refactoring is gathered.  %The most frequent modification to the KDM instances in this scenario will be, intuitively, creating new metaclasses. However, updating existing metaclasses with their relationship will be frequent as well. In simpler cases, updating means changing properties of existing metaclasses. In more complex cases, updating means removing metaclasses and replacing them with new and refactored ones. This step uses model-to-model (M2M) transformation language to perform the refactorings.

%In the step [B], \textit{Mine Affected Metaclasses}, we developed a mechanism which shows all metaclasses that need to be updated/propagated after applying any changes/refactoring. These metaclasses are those that have some dependence on the metaclass to be modified by the refactoring. This step is totally based on a set of queries that works on a KDM instance. In addition, this step uses depth-first search algorithm\footnote{From here on in Dependents Identification Algorithm (DI Algorithm)} to identify all affected metaclasses along with a set of queries. 

%In step [C], \textit{Propagate Changes}, involves updating the elements identified in the step [B].  As in step [A], in this step we also have used M2M to update/propagate all KDM's instances. 

\subsection{Identifying Diff Between Models}

The problem of model diff is intrinsically complex. For instance, if a ClassUnit C is deleted, its transitive parts and attached associations are typically also deleted. So, computing the difference results
in a large number of detail changes that might be complex to implement. Therefore, in this step we have used the EMF Compare\footnote{https://www.eclipse.org/emf/compare/}, which is a framework that can easily reuse and extend to compare instances of any models, in our case KDM models. We have chosen to use EMF Compare because it was designed with scalability in mind in order to support comparisons of large fragmented models.

To start this step, our modified EMF Compare needs two instance of KDM as input: i) the refactored one (left side), and ii) the original (right side). The EMF Compare analyses, whether the KDM elements are equal or if they present differences (for example, the name of the class has been changed from Class1 to ClassX, or a class has been moved from a Package to another Package, etc). EMF Compare iterates over all of our KDM elements, be they unmatched (only one side has this object), couples (two of the three sides contain this object) and compute any difference that may appear between the sides. 
For example, a KDM element that is only on one side of the comparison is a KDM element that has been added, or deleted. But a couple might also represent a deletion: during three way comparisons, if we have an KDM element in the common ancestor (origin) and in the left side, but not in the right side, then it has been deleted from the right side version. The output of this step is a list that contains all KDM model elements involved during the KDM refactoring.

\subsection{Identifying Points to Propagate} % (fold)
\label{sub:mine_affected_metaclasses}

The step [B] starts with DI Algorithm to identify all affected meta-classes along with a set of queries. The DI Algorithm that aims to identify all meta-classes and its relationships that use somehow the meta-class(es) that were refactored. As input the list obtained from the EMF Compare in Step [A] is needed. For example, in the case of the refactoring \textit{Move Class} the list would contains a package and a set of classes that were moved. Further, our DI Algorithm uses a set of queries that are performed on the KDM's instance to mine all the affected/linked meta-classes. All the queries were created using XPath. We have decided to use XPath because it is a well-know and well-documented language. 

%Concerning the refactoring \textit{Move Class} the engineer should specify a set of classes that no longer are contained into a package \texttt{View}. These classes should be allocated into the package \texttt{Model}. 
%Considering the refactoring \textit{Move Class}, three elements (\texttt{Student}, \texttt{Instructor}, and their package) need to be investigated throughout the KDM's instance in order to identify which other metaclasses can be affected. 

Firstly a query must be executed to get the root element in KDM. This query is represented as the first statement in Figure~\ref{fig:queriesXPath}, see line 1 - it is used to return an instance of the meta-class \texttt{Segment}. The returned Segment, as well as all KDM's levels are gathered by the other queries presented in Figure~\ref{fig:queriesXPath} lines 2 to 5. The returned elements of these queries are used as input in our DI Algorithm as all the the list obtained from the Step [A].

\begin{figure}[h]
	\centering
	% Requires \usepackage{graphicx}
	\includegraphics[scale=0.479]{figuras/queiresANDATLSBESNew}
	\caption{Xpath used to return the KDM's root element, Segment.}
	\label{fig:queriesXPath}
\end{figure}


\begin{algorithm}[h]
     \SetAlgoLined
     \KwIn{DFS (G, u, eL) where G is a KDM's instance, u is the initial meta-class, i.e., Segment, and eL is a set of elements to verify}
     \KwOut{A collection of affected meta-classes}
     \Begin{
     \ForEach{$outgoing$ edge e = (u, v) of u} {
	\If{vertex v as has not been visited }{
			\If{vertex v contain implementation = true }{
				
				\ForEach{$implementations$ element}{
				verify all elements in implementation
				}
				Mark vertex v as visited (via edge e).
				Recursively call DFS (G, v).
			}
			
				}				
			}		
	
	}
     \caption{DFS(G,u) - Depth-First Search Algorithm.}
     \label{alg:death1}
   \end{algorithm}

Algorithm~\ref{alg:death1} depicts the DI Algorithm that is used to mine all the affected meta-classes. %It takes as input a KDM's instance, a \texttt{Segment}, and a set of elements that were refactored in Step [A] (e.g., for the refactoring \textit{Move Class} three affected elements - \texttt{Student}, \texttt{Instructor}, and their package). %A diagram of how our DI Algorithm works is shown in Figure~\ref{fig:algWorks2}. Each node represents a metaclass and the edges represent the relationship among the metaclasses - the node A represents the \texttt{Segment} and K, H, E and B illustrate \texttt{CodeModel}, \texttt{StructureModel}, \texttt{ConceptualModel}, and \texttt{DataModel}, respectively. 
%
More specifically, the algorithm works as follows: first it is necessary to pick a starting point, i.e., the meta-class \texttt{Segment}. Visit the \texttt{Segment}, push it onto a stack, and mark it as visited. Then it is necessary to go to the next meta-class that is unvisited, verify if it has an association named \texttt{implementation}. If yes, it verifies if this association contains references to any element's used in the refactoring, if yes - push it on the stack, and mark it. This continues until the algorithm reaches the last meta-class. Then the algorithm checks to see if the \texttt{Segment} has any unvisited adjacent meta-class. If it does not, then it is necessary to pop it off the stack and check the next meta-class. If the algorithm finds one (unvisited meta-class), it starts visiting adjacent meta-classes until there are no more, check for more unvisited adjacent meta-classes, and continue the process always verifying the association named \texttt{implementation}. When the algorithm finally reach the last meta-class on the stack and there are no more adjacent, unvisited meta-classes that contains the association \texttt{implementation} without check, our algorithm should create a list of all affected meta-classes that is further used to propagated all changes throughout the KDM levels. 
%
%
%
%
%
%
%
%
%

%\begin{figure}
%	\centering
	% Requires \usepackage{graphicx}
%	\includegraphics[scale=0.2]{figuras/algWorks2}
%	\caption{Depth-First Search.}
%	\label{fig:algWorks2}
%\end{figure}






%As can be visualized, a stack is used to store all affected elements, see Algorithm~\ref{alg:death1} line 2, \ding{182}. In line 3 a generic KDM element is defined. While \textit{seg} is non-empty, a node is chosen for expansion (line 4). %For fact edges, the dependency of the edge on the particular fact that caused its creation is then recorded (lines 14- 16)





%\begin{algorithm}[h]
%     \SetAlgoLined
%     \KwIn{KDMEntity kdmElement, Segment segment, KDMModel model}
%     \KwOut{All affected metaclasses}
%     \Begin{
%   $ Stack stack \longleftarrow \{\}$\;
%   KDMEntity elementToVerify\;
%     \ForEach{$seg$ in $segment$} {
%	\eIf{seg.getOwnedElements() != null}{
%			\If{seg.getNextSiblind() != null}{
%				$elementToVerify \longleftarrow seg.getNextSiblind()$\;
%		\If{\ding{182} isAffected(elementToVerify, kdmElement, model)}{
%					stack.push(elementToVerify)\;
%					$seg \longleftarrow  seg.getFirstChild()$\;
%				}				
%			}		
%	
%	}{ $seg \longleftarrow seg.getNextSiblind()$\;
%		\If{seg = null \&\& stack.isEmpty()}{
%			// return to the parent's level
%		}
%	}
%     }
%	\ding{184} \Return{stack}
%     }
%     \caption{Depth-First Search Algorithm.}
%     \label{alg:death1}
%   \end{algorithm}
%
%\begin{algorithm}[h]
%     \SetAlgoLined
%     \KwIn{KDMEntity kdmElement, KDMModel model, KDMEntity e}
%     \KwOut{true or false}
%     \Begin{
%     	\If{ (e = AbstractUIElement) or (e = AbstractStructureElement) or (e = BuildResource) or (e = AbstractPlatformElement) or (e = AbstractConceptualElement) or (e = AbstractEventElement)
%				} {
%					     \ForEach{$elements$ in $e.getImplementation()$} {
%					\If{ elements = ele} {
%					\Return{true}
%					}
%					}				
%				}
%\uElseIf{ e = AbstractDataElement
%				} {
%					     \ForEach{$elements$ in $elementToV.getDataRelation()$} {
%					\If{ elements = elementToVerify} {
%					\Return{true}
%					}
%					}				
%				}
%	...
%     }
%     \caption{isAffected Algorithm}
%     \label{alg:death}
%   \end{algorithm}

%As every element, except the Segment, is connected somehow it is necessary to iterate throughout them, line 4 of Algorithm~\ref{alg:death} illustrates this iteration. After, the method \texttt{isAffected(...)} is called to verify if the element is affected.  If the condition in line 8 evaluates to true, then the element is pushed into the stack defined in line 2. Finally, in line 20 the stack is returned with all affected elements, see Algorithm~\ref{alg:death} \ding{184}. 



\subsection{Performing Propagation} % (fold)
\label{sub:apply_refactoring}

This step is a decoupled module that can be coupled to existing refactorings. In this way, existing users can write KDM refactorings in ATL without worrying about the change propagation, which is a time-consuming and error-prone task. The only task is to provide for our component all the parameters it needs to conduct the propagation. In our approach these parameters are identified automatically in Step [B] and are used in Step [C]. 
%
%
 %all propagations regarding an specific refactoring, e.g., \textit{Move Class}, are implemented. 
Similarly to the step [A], where the modernizer has to define a set of model transformations rules to perform the model refactoring, here a set of generic and pre-established model transformations (written in ATL) are used. The difference is that in Step [A], the modernizer can either create or reuse a KDM refactoring, otherwise in Step [C] all rules were beforehand defined to perform the propagation of changes (in a cascade way) after the application of a KDM refactoring. In addition, these ATL rules (the propagations) require a set of  \textit{mininum} parameters that should be informed before realize all the propagations. As already mentioned these parameters are the output from Step [B], which is a list containing all KDM affected elements. 

In order to bound these parameters along with the output of Step [B] our approach performs a static analysis (parsing) of all generic ATL rules and identifies places that must be replaced in with the Step [B]'s output (KDM affected elements), i.e, all the places where parameters are needed. This is particularly necessary in our approach because ATL does not enforce type correctness, hence rules written in ATL may be ill-typed. Moreover, the creation of a suitable propagation %in this step 
requires precise parameters (meta-classes) informations. It is important to highlight that this static analysis is done totally automatically and transparently by means of our Eclipse plug-in. The aim is having a decouple module of KDM propagation as simple as possible to facilitate the integration with any refactoring defined in ATL in the context of KDM model and also to promote the reuse. Therefore, the software modernizer does not have to worry about devising the propagation of changes, which usually is harder than just the creating of a KDM refactoring. 
%
In addition, if the static analysis detect errors, the software modernizer is required to fix and inform the correct parameters, otherwise, all changes are propagated in all KDM levels automatically/transparently  


Figure~\ref{fig:ATLPropagation} shows a code snippet written in ATL that is used to propagate the changes. Due space limitation the whole ATL it is not presented. Note that all strings, \textbf{`\#parameter'}, are changed during the static analysis along with the step [B]'s output. As can be seen, there are three rules - each of them is used to propagated the change in a specific KDM package, respectively. The first rule is responsible to propagate the changes throughout the \texttt{Structure Package}, see lines 24 to 32. In line 26 the source pattern of the rules is defined by using OCL guard stating the layers to be matched. After, is defined a target pattern (lines 29 -31) which is used to compute the \texttt{density} of an \texttt{AggregationRelationship} after the application of a refactoring, i.e, \textit{Move Class}.

\begin{figure}[h]	
	\centering
	% Requires \usepackage{graphicx}
	\includegraphics[scale=0.516]{figuras/ATLPRopagationSBESFormatted}
	\caption{Chunk of code in ATL to perform the propagation after the application of refactoring \textit{Move Class}.}
	\label{fig:ATLPropagation}
\end{figure}

If the \textit{Move Classes} refactoring is applied to transfer the class C1 to package P2, a natural propagation is to transfer the business rule B1 to another scenario. As defined in the second rule (lines 33 to 43) - this rule is used to propagate the changes throughout the \texttt{Conceptual Package}.
%
%The rule defined in lines 33 to 43 propagates the changes throughout the \texttt{Conceptual Package}. If the \textit{Move Classes} refactoring is applied to transfer the class C1 to package P2, a natural propagation is to transfer the business rule B1 to another scenario.
%
%
% For instance, Line 36 shows the specific propagation that is ne the \texttt{RuleUnit 1.1} that is associated with \texttt{Instructor} should also be moved to corresponding scenario, i.e, the scenario that is associated with the package that contains now the class \texttt{Instructor} - \texttt{ScenarioUnit 3}. 
%
 Finally, the rule defined in lines 44 - 65 aims to propagate the change to the \texttt{Data Package}. For each \texttt{ClassUnit}, a \texttt{RelationalTable} instance has to be created - their names have to correspond. The \texttt{itemUnit} %(a collection that contains \texttt{ColumnSet}) 
reference set has to contain all \texttt{ColumnSet} that have been created for each \texttt{StorableUnit} (meta-class that represent all the attributes that a class holds) as well as its types.

%In the step [C] all the changes, that where resulted by the refactoring  performed in step [b] need be propagate into all KDM's levels/package.  

%The correctness of our Propagation Engine mainly relies on the compliance of the transformation result, which in turn can be ensured by showing that the transformation produces a consistent KDM model. Thus, it is important to support the consistency among all KDM views. To this end, we consider static consistency checks.

Although we have used a simple refactoring as example, by observing both Figure~\ref{fig:ATLRefactoring} and Figure~\ref{fig:ATLPropagation} one can notice that the refactoring itself usually is less complex/verbose to devise than the Propagation Engine, i.e., a set of rules defined to propagate the changes in KDM levels tend to be more complex/verbose than KDM refactorings. Therefore, providing a module that can be plugged over existing KDM refactorings in order to propagate the changes can assist the software modernizer.

% Due space limitation the whole ATL that is used to promote the propagation is not shown - but by observing both ATL that the rules to perform the propagation are much more complex. 

%\section{Refactoring Meta-model}~\label{sec:refactoring_metamodel}
%	%As stated beforeKDM does not provides suitable meta-classes to apply refactorings. 

Previous work on model refactoring focused on UML models (Biermann et al., 2006; Mens, 2005; Mens et al., 2007). However, when these UML models are refactored, the respective changes have to be propagated into either different artefacts or other levels, i.e., distinct models; otherwise, the system represented in model is no longer consistent with the model. As the KDM is a meta-model that aims to model all of a given system, it allows the propagation and traceability of changes after the application of a refactoring.


Previous works are related to static context - within a static context, refactoring of model will probably never cause any problems. But many models evolve in time: elements are renamed, layer are created, relationship are changed, references are re-targeted and classes are inlined or extracted in order to create or collapse inheritance hierarchies or just to improve the model - in short: the model is refactored. Regarding only UML models, applying these refactorings is well understood and implemented in several ways. However, in the context of the KDM models there is none research in this area. Therefore, we claim that research must to be done in this direction. The problem is usually caused by the strong interconnection between KDM's packages, all KDM's packages are connected somehow (see Section~\ref{sec:background}), e.g., the package Structure uses meta-classes from package Code, the package Code uses meta-classes from package core, etc. Therefore, if any meta-class is just referencing the refactored model meta-class, after refactoring, this reference can be unset. Hence, the KDM model is invalid because of inconsistencies between its packages. 


Therefore in this section we propose an approach to assist the propagation of refactoring into KDM's packages. Our approach starts with Fowler`s definition of each refactoring. As these definitions were introduced for the refactoring of (object oriented) code, we adapt the definitions for KDM model refactorings. Subsequently we investigate how the changes affect the KDM's packages. Herein we consider different aspects concerning the KDM's packages. For instance, depending on the meta-class the refactoring may cause minor or major changes to be propagated into other packages.

In order to explain the propagation of changes in KDM in this section a set of examples is used. The first system is presented in Figure~\ref{fig:system} (A).As noted in Figure~\ref{fig:system} (A) two layers have been defined. The first one is ``View'' and the second layer is ``Controller''. Inside of each layer there is one package. The relationships ``GUI inside View'' and ``CTR inside Controller'' mean that ``GUI'' and ``CTR'' are in container ``View'' and ``Controller'', respectively or in some sub-container of  ``View'' and ``Controller'', transitively. In the same way, the relationship ``StudentGUI inside GUI'' means that ``StudentGUI'' is in container ``GUI'' or in some sub-container of ``GUI''. For relationships, let R' be the corresponding aggregated relationship, which represents the number summing all primitive relationships, i.e., ``Calls'', ``Creates'', ``Extends'', etc. 

The corresponding, though simplified KDM instance of Figure~\ref{fig:system} (A) is depicted in Figure~\ref{fig:system} (B). It illustrates a KDM instance as a UML object diagram for the sake of simplicity, note that this diagram represents the system in  as a tree of nodes containing some KDM`s meta-classes. Analysing both figures it is evident that each element presented in Figure~\ref{fig:system} (A) has a meta-class in KDM to represent it. For instance, the layers are represented in KDM using the meta-class Layer. Each primitive relationship has also a meta-class in KDM. The meta-class ``AggregatedRelationship'' represents a set of primitive KDM relationships. 

In order to increase the modularity of the system consider that the engineer has chosen to apply refactoring extract package to create a package model. In Figure~\ref{fig:atl} is depicted a chunk of code written in ATL responsible to perform the refactoring extract package.

Firstly, an instance of meta-class Package must be created. Then the name of it must be defined. After the creation of this package one must group the instance of all classes that will be moved to the new package. The refactoring at this point can be considered complete for the package code of KDM. However, now it is necessary to propagate all the changes to other KDM's packages. In Figure~\ref{fig:atl2} is presented a chunk of code responsible for performing the propagation of changes. As can be seen, it is necessary to create an instance of the meta-class Layer, set its name to ``Controller'', see lines X-Y of Figure~\ref{fig:atl2}. Then, the association ``implementation'' of ``Layer'' earlier created need to be specified. In this point it is important to visualise that this association refers to the extracted package, i.e., the package created earlier. Further, it is necessary to create an instance of meta-class AggregatedRelationship. This meta-class has a meta-attribute, ``density'' and a set of association that need to be propagated. The part of the source code responsible to create an AggregatedRelationship is presented in lines 17-23 of Figure~\ref{fig:atl2}. The helper illustrated in lines X-Y represents that the meta-attribute ``density'' is updated always that a new relationship is identified.




We can divide the propagation in three the possible scenarios. The first scenario is presented in Figure X. As noted 



Suppose the system presented in Figure~\ref{fig:system}. This system was developed using the architectural style Model-View-Controller.



\section{Case Study}\label{sec:case_study}
	\linespread{0.87}
	%!TEX root = /Users/rafaeldurelli/Dropbox/Artigos Elaborados/KDM propagation_2015/sbes_2015_kdm_propagation/sbes2015_kdm_propagation.tex

\section{Case Study}

In this section we present a case study showing that our approach can be used to support the change propagation in KDM models. We have used a real-life legacy information system. 
Notice that the case study was carried out following the protocol for planning, conducting and reporting case studies proposed by Brereton et al. in [17] improving the rigor and validity of the study. The next subsections show more details about the main phase defined in this protocol, such as: background, design, case selection, case study procedure, data collection, analysis and interpretation and validity evaluation.

\subsection{Research Question}

According to the protocol proposed by Brereton et al. in~\cite{Brereton:2008} firstly it is needed to identify previous research on the topic. Hence, in Section~\ref{sec:related_work} we stated some researches related to refactoring in models. However, not of them are using ADM/KDM. We are particularly focus on the propagation of changing in different views of a KDM model. In this context, the object of this study is the proposed approach, and the purpose of this study is the evaluation of the approach herein described related to its effectiveness and efficiency.

Therefore, taking into account the object and purpose of the study, it was defined one research question, as follows:

\begin{itemize}
\item \textbf{RQ$_1$}: Given a set of refactoring, can the proposed approach propagate all the changes effectively throughout all KDM levels?
\end{itemize}


\subsection{Design}

The described case study consist of a single case~\cite{Brereton:2008}. It was focused on a single legacy system. To assess the effectiveness of the proposed approach through the \textbf{RQ$_1$}, we use some oracles. As each refactoring has its own characteristics and modifies specific model elements, it is possible to predict all the expected changes in other KDM levels. So, considering our set of developed refactorings, we had to develop some oracles for each refactoring. The complete oracle can be see at www.mudar.com.br.

\subsection{Case Selection}

In this section is described the suitable case that was chosen to be studied. Some criteria were applied to select the suitable case, as follows: (i) it must be an enterprise system, (ii) it must be a Java-based system, (iii) it must be a legacy system and (iv) it must be of a size not less than 10 KLOC. After applying these criteria we chose LabSys (Laboratory System) it is currently used by Federal University of Tocantins (UFT). It is used to control the use of laboratories in the entire university. 

\subsection{Case Study Procedure}\label{sec:caseStudyProcedure}

In this section is shown how the execution of the study was planned. Notice that the execution was aided by the tool developed to support the proposed approach. The case study was carried out in a machine with an Intel Core I5 CPU 2.5GHz, 8GB of physical memory running Mac OS X 10.8.4.

The proposed approach uses as initial artefact a KDM instance,. Therefore, firstly we adopted a reverse engineering to  transform the LabSystem source-code into a KDM instance to apply our approach. In this step we have used MoDisco\cite{Brunele20141012}, which is a parser that get as input java source-code and then return as output a KDM instance. Currently, MoDisco only generates the KDM \texttt{Code package}, other KDM packages are extremely important to evaluate our approach. Therefore, we have manually instantiated the followings KDM packages: \texttt{Structure Package}, \texttt{Data Package}, and \texttt{Conceptual Package}.


We selected for refactorings for our evaluation: \textit{Extract Class}, \textit{Move Class}, \textit{Extract Layer} and \textit{Remove Class}. %We applied each of the four refactorings to every possible location in KDM instance. 
It is worth to notice that all refactorings were applied completely automatically by means of our devised proof-of-concept tool. To deal with refactorings that go into infinite loops, we set three minutes timeout interval. More specifically, we applied the \textit{Extract Class} to one class that had more than 300 LOC (Line of Code); we applied the \textit{Move Class} to a set of  class from a package to another package; we applied the \textit{Extract Layer} to a layer that contains at least 20 classes; finally we applied the \textit{Remove Class} to a class that contained at least 15 primitive relationships. After applied all refactorings we verify whether them were successful, i.e., if the intended refactoring could be performed, and if all the expected propagations were generated on the model. 

\subsection{Data Collection and Interpretation}

We verify, based on a set of oracle, whether all refactoring were successfully propagated throughout all KDM models. By using these information gathered we can draw conclusion and answer the \textbf{RQ$_1$}.

\begin{table}
\caption{Propagation - Extract Class}
\label{table:propagationExtractClass}
{\footnotesize{}}%
\setlength{\tabcolsep}{0.0em}
{\renewcommand{\arraystretch}{0.5}
\begin{tabular}{|>{\centering}p{2cm}|>{\raggedright}m{5cm}|>{\centering}p{4cc}|}
\hline 
{\footnotesize{Refactoring}} & {\footnotesize{Extract Class}} & {\footnotesize{P. C?}}\tabularnewline
\hline 
\hline 
\multirow{4}{2cm}{{\footnotesize{Code Package}}} & {\footnotesize{Create an instance of ClassUnit that represent the
new Class}} & {\footnotesize{Yes}}\tabularnewline
\cline{2-3} 
 & {\footnotesize{Move all StorableUnits to the new ClassUnit}} & {\footnotesize{Yes}}\tabularnewline
\cline{2-3} 
 & {\footnotesize{Move all MethodUnit to the new ClassUnit}} & {\footnotesize{Yes}}\tabularnewline
\cline{2-3} 
 & {\footnotesize{Create an intance of HasType, which represent an association
between the new ClassUnit and the old ClassUnit}} & {\footnotesize{Yes}}\tabularnewline
\hline 
{\footnotesize{Structure Package}} & {\footnotesize{Not Applied}} & {\footnotesize{Not Applied}}\tabularnewline
\hline 
\multirow{3}{2cm}{{\footnotesize{Data Package}}} & {\footnotesize{Create a instance of RelationalTable owning the name
of the new ClassUnit}} & {\footnotesize{Yes}}\tabularnewline
\cline{2-3} 
 & {\footnotesize{For each StorableUnit it is necessary to create a ItemUnit,
which represent the RelationaTable columns.}} & {\footnotesize{Yes}}\tabularnewline
\cline{2-3} 
 & {\footnotesize{Create an instance of UniqueKey that represent the
primary key of the RelationalTable.}} & {\footnotesize{Yes}}\tabularnewline
\hline 
{\footnotesize{Conceptual Package}} & {\footnotesize{Not Applied}} & {\footnotesize{Not Applied}}\tabularnewline
\hline 
\end{tabular}}
\end{table}
\begin{table}
\caption{Propagation - Move Class}
{\footnotesize{}}%
\setlength{\tabcolsep}{0.0em}
{\renewcommand{\arraystretch}{0.5}
\begin{tabular}{|>{\centering}p{2cm}|>{\raggedright}p{5cm}|>{\centering}p{4cc}|}
\hline 
{\footnotesize{Refactoring}} & {\footnotesize{Move Class}} & {\footnotesize{P.C?}}\tabularnewline
\hline 
\hline 
{\footnotesize{Code Package}} & {\footnotesize{Move an specific ClassUnti from a source Package to
a target Package}} & {\footnotesize{Yes}}\tabularnewline
\hline 
{\footnotesize{Structure Package}} & {\footnotesize{If the target Package is associated to an architectural
elements by means of the association implementation the value of meta-attribute
named density should be propagated }} & {\footnotesize{Yes}}\tabularnewline
\hline 
{\footnotesize{Data Package}} & {\footnotesize{Not applied}} & {\footnotesize{Not applied}}\tabularnewline
\hline 
{\footnotesize{Conceptual Package}} & {\footnotesize{If the moved class is associated to any conceptual
elements by means of the association implementation this conceptual
elements should be moved to a correspondent associated element of
the target Package. }} & {\footnotesize{Yes}}\tabularnewline
\hline 
\end{tabular}}
\end{table}
\begin{table}
\caption{Propagation - Extract Layer}
{\footnotesize{}}%
\setlength{\tabcolsep}{0.0em}
{\renewcommand{\arraystretch}{0.5}
\begin{tabular}{|>{\centering}p{2cm}|>{\raggedright}p{5cm}|>{\centering}p{4cc}|}
\hline 
{\footnotesize{Refactoring}} & {\footnotesize{Extract Layer}} & {\footnotesize{P.C?}}\tabularnewline
\hline 
\hline 
\multirow{2}{2cm}{{\footnotesize{Code Package}}} & {\footnotesize{Create an instance of Package}} & {\footnotesize{Yes}}\tabularnewline
\cline{2-3} 
 & {\footnotesize{Move a the selected ClassUnit from a Package to the
new Package}} & {\footnotesize{Yes}}\tabularnewline
\hline 
\multirow{4}{2cm}{{\footnotesize{Structure Package}}} & {\footnotesize{Create an instance of Layer }} & {\footnotesize{Yes}}\tabularnewline
\cline{2-3} 
 & {\footnotesize{Create an instance of AggregationRelationship between
the new Layer and the old one}} & {\footnotesize{Yes}}\tabularnewline
\cline{2-3} 
 & {\footnotesize{Associate the new Layer by means of the association
implementation with the new Package}} & {\footnotesize{Yes}}\tabularnewline
\cline{2-3} 
 & {\footnotesize{Summing up all primitive relationship to compute the
meta-attribute density}} & {\footnotesize{Yes}}\tabularnewline
\hline 
{\footnotesize{Data Package}} & {\footnotesize{Not applied}} & {\footnotesize{Not applied}}\tabularnewline
\hline 
{\footnotesize{Conceptual Package}} & {\footnotesize{If the moved ClassUnits are associated to any conceptual
elements by means of the association implementation these conceptual
elements should be moved to a correspondent associated element of
the target Package. }} & {\footnotesize{Yes}}\tabularnewline
\hline 
\end{tabular}}
\end{table}
\begin{table}
\caption{Propagation - Remove Class}
\label{tab:propagationRemoveClass}
{\footnotesize{}}%
\setlength{\tabcolsep}{0.0em}
{\renewcommand{\arraystretch}{0.5}
\begin{tabular}{|>{\centering}p{2cm}|>{\raggedright}p{5cm}|>{\centering}p{4cc}|}
\hline 
{\footnotesize{Refactoring}} & {\footnotesize{Remove Class}} & {\footnotesize{P.C?}}\tabularnewline
\hline 
\hline 
{\footnotesize{Code Package}} & {\footnotesize{Delete the selected instance of a ClassUnit}} & {\footnotesize{Yes}}\tabularnewline
\hline 
{\footnotesize{Structure Package}} & {\footnotesize{If the removed ClassUnit was contained into a specific
Structure element then summing up all primitive relationship and overwrite
the meta-attribute density}} & {\footnotesize{Yes}}\tabularnewline
\hline 
{\footnotesize{Data Package}} & {\footnotesize{If the removed ClassUnit was associated with an instance
of RelationalTable, then it should also be removed}} & {\footnotesize{Yes}}\tabularnewline
\hline 
{\footnotesize{Conceptual Package}} & {\footnotesize{If the removed ClassUnit is associated to any conceptual
elements by means of the association implementation these conceptual
elements should be removed }} & {\footnotesize{Yes}}\tabularnewline
\hline 
\end{tabular}}
\end{table}


%\begin{table}[h]
%\centering
%\caption{Propagation Result\label{tab:propagation}}
%~~\\
%\begin{tabularx}{
%.30\textwidth}{|c|X|}
%There is no difference between using our tool and using an ad-hoc reuse process in terms of productivity (time) to couple sucessfully a CF with an application.
%\hline \cellcolor[gray]{\shadow} Refactoring & \footnotesize{Propagation Corrected?}
%\\
%\hline \cellcolor[gray]{\shadow} Extract Class & \footnotesize{Yes}
%\\
%\hline \cellcolor[gray]{\shadow} Move Class & \footnotesize{Yes}
%\\
%\hline \cellcolor[gray]{\shadow} Extract Layer & \footnotesize{Yes}
%\\
%\hline \cellcolor[gray]{\shadow} Remove Class & \footnotesize{Yes}
%\\
%\hline
%\end{tabularx}
%\end{table}

Tables~\ref{table:propagationExtractClass} to~\ref{tab:propagationRemoveClass} summaries the results related to each refactoring applied and its respective propagations. ``P.C?" stands for ``Propagation Corrected?" As can be seen the all the changes were effectively propagated throughout all KDM levels. Which means that in this case our approach could automatically execute truly relevant propagation throughout all KDM levels when dealing with the refactorings: \textit{Extract Class}, \textit{Move Class}, \textit{Extract Layer} and \textit{Remove Class}. Thereby, the \textbf{RQ$_1$} can be answered as true, that is, the proposed approach can propagate changes effectively throughout all KDM levels.

%%!TEX root = /Users/rafaeldurelli/Dropbox/Artigos Elaborados/KDM propagation_2015/sbes_2015_kdm_propagation/sbes2015_kdm_propagation.tex
\section{Proof-of-Concept Implementation}

We devised a Eclipse plug-in named Modernization-Integrated Environment (MIE) which is split in three layers, as follows: (\textit{i}) Core Framework, (\textit{ii}) Tool Core, and (\textit{iii}) Graphical User Interface (GUI). This plugin was devised on the top of the Eclipse Platform; The first layer we used both Java and Groovy as programming language. Moreover, the Core Framework layer contains a set of Eclipse plug-ins on which our environment is based on, such as MoDisco and Eclipse Modeling Framework (EMF)\footnote{http://www.eclipse.org/modeling/emf/}. We used MoDisco\footnote{http://www.eclipse.org/MoDisco/} once it is an extensible framework to develop model-driven tools to support use-cases of existing software modernization and provides an Application Programming Interface - (API) to easily access the KDM model. Also, EMF was used to load and navigate KDM models that were generated with MoDisco. The second layer, the Tool Core, is where the steps presented in Section~\ref{sec:the_approach} were implemented. Herein, we work intensively with KDM models, which are XML files. Therefore, we use XPath to handle those types of files, to mine the affected metaclasses, ATL to perform the refactoring and to propagated them. Finally, the third layer is the Graphical User Interface (GUI) that consists of a set of SWT windows with several options to perform the refactorings based on the KDM model.




% Figure~\ref{fig:architecture} depicts the architecture of this environment. As shown in this figure, the first layer is the Core Framework. This layer represents that  we devised the  environment on the top of the Eclipse Platform. In this layer it is also possible to see that we used both Java and Groovy as programming language. Moreover, this layer contains Eclipse Plugins on which our environment is based on, such as MoDisco and EMF. We used MoDisco\footnote{http://www.eclipse.org/MoDisco/} once it is an extensible framework to develop model-driven tools to support use-cases of existing software modernization and provides an Application Programming Interface - (API) to easily access the KDM model. Also, Eclipse Modeling Framework (EMF)\footnote{http://www.eclipse.org/modeling/emf/} was used to load and navigate KDM models that were generated with MoDisco. 

%\begin{figure}[!ht]
%\centering
  % Requires \usepackage{graphicx}
 % \includegraphics[scale=0.6]{FIGURAS_DA_REFATORACAO/Arquitetura}
%\caption{Architecture of the proof-of-concept implementation.}
%\label{fig:architecture}
%\end{figure}  

%The second layer, the Tool Core, is where all refactorings provided by our environment were implemented. It works intensively with KDM models, which are XML files. Therefore, we use Groovy to handle those types of files because of the simplicity of its syntax and fully integrated with Java.  After the engineer to realize all refactorings a forward engineering must be carried out - then the source code of the refactored target system is generated. Finally, the top layer is the Graphical User Interface (GUI) that consists of a set of SWT windows with several options to perform the refactorings based on the KDM model.
\section{Evaluation}\label{sec:evaluation}
	\linespread{0.87}
	%!TEX root = /Users/rafaeldurelli/Dropbox/Artigos Elaborados/KDM propagation_2015/sbes_2015_kdm_propagation/sbes2015_kdm_propagation.tex

\section{Evaluation}\label{sec:evaluation}

This section describes the experiment used to evaluate the change propagation effectiveness of our approach. In fact, two experiment were conducted. The first experiment is called ''Mining Study`` and was planned to identify the effectiveness of our  mining algorithm. Therefore, we have compared its result with an oracle in order to verify its correctness. The second experiment is referred as ``Propagation Study'' and was planned to evaluated the correctness of the propagation given a set of refactorings. In addition, we have worked out two research questions, as follows:

%Moreover, this experiment also evaluate the devised Eclipse plug-in, which was earlier described. Specifically, we investigate the following research questions:

\textbf{RQ$_{1}$}: Given some specific elements to be refactored, is the mining algorithm able to identify correctly all the dependent KDM elements?

\textbf{RQ$_{2}$}: Given a specific refactoring R, are all dependent elements identified in the oracle correctly refactored?
 
%To evaluate these questions we we carried out two steps. Firstly, we have evaluated our mining algorithm. Therefore, we have compared its result with an oracle in order to verify its correctness. Similarly, we have evaluated the correctness of a set of refactorings. Thus,  we have also compared its results to the same oracle mentioned previously.

\subsection{Goal Definiton}\label{sec:goal_definition}

We use the organization proposed by the Goal/Question/Metric (GQM) paradigm, it describes experimental goals in five parts, as follows:
%
%\begin{itemize}
%
%\item \textbf{object of study:} the object of study is our approach; 
%
%\item \textbf{purpose:} the purpose of this experiment is to evaluate the effectiveness of our approach, i.e, our mining affected metaclasses and the propagation of changes;
%
%\item \textbf{perspective:} this experiment is run from the standpoint of a researcher;
%
%\item \textbf{quality focus:} the primary effect under investigation is the precision and recall after applying the mining algorithm and a set of refactorings; 
%
%\item  \textbf{context:} this experiment was carried out using Eclipse 4.3.2 on a 2.5 GHz Intel Core i5 with 8GB of physical memory running Mac OS X 10.9.2.
%\end{itemize}
%
%
(\textit{i}) \textbf{object of study:} the object of study is our approach; (\textit{ii}) \textbf{purpose:} the purpose of this experiment is to evaluate the effectiveness of our approach, i.e, our mining affected metaclasses and the propagation of changes; (\textit{iii}) \textbf{perspective:} this experiment is run from the standpoint of a researcher; (\textit{iv}) \textbf{quality focus:} the primary effect under investigation is the precision and recall after applying the mining algorithm and a set of refactorings; (\textit{v}) \textbf{context:} this experiment was carried out using Eclipse 4.3.2 on a 2.5 GHz Intel Core i5 with 8GB of physical memory running Mac OS X 10.9.2.
The experiment can be defined as: \textbf{Analyze} the effectiveness of both the change propagating of our approach and the mining affected metaclasses, \textbf{for the purpose of} evaluation, \textbf{with respect to} precision and recall, \textbf{from the point of view of} the researcher, \textbf{in the context of} a subject program. 

\subsection{Hypothesis Formulation}\label{hypothesis_formulation}	
In order to accomplish our goal, we explored the formalization of our research questions into the following hypotheses:

%Our research questions were formalized into hypotheses so that statistical tests can be performed. 

\begin{table}[h]
\centering
\caption{Hypotheses for the Mining Study\label{tab:hypotheses}}
~~\\
\begin{tabularx}{
.46\textwidth}{|c|X|}
%There is no difference between using our tool and using an ad-hoc reuse process in terms of productivity (time) to couple sucessfully a CF with an application.
\hline \cellcolor[gray]{\shadow} H$_0$ & \footnotesize{ There is no difference in pattern recognition before and after to apply our mining affected metaclasses algorithm into the KDM model (measured in terms of the metric precision (P) and recall (R)) which can be formalized as: 

\textbf{H$_{0}$: $\mu_{P_{Bf}} = \mu_{P_{Af}}$ and $\mu_{R_{Bf}} = \mu_{R_{Af}}$}}
\\
\hline \cellcolor[gray]{\shadow} H$_1$ & \footnotesize{There is a significant difference in pattern recognition before and after to apply our mining affected metaclasses algorithm into the KDM model (measured in terms of the metric precision (P) and recall (R)) which can be formalized as: 

\textbf{H$_{1}$: $\mu_{P_{Bf}} \neq \mu_{P_{Af}}$ and $\mu_{R_{Bf}} \neq \mu_{R_{Af}}$}}
\\
\hline
\end{tabularx}
\end{table}

\begin{table}[h]
\centering
\caption{Hypotheses for the Propagation Study\label{tab:hypotheses}}
~~\\
\begin{tabularx}{
.46\textwidth}{|c|X|}
%There is no difference between using our tool and using an ad-hoc reuse process in terms of productivity (time) to couple sucessfully a CF with an application.
\hline \cellcolor[gray]{\shadow} H$_0$ & \footnotesize{ There is no difference in propagation of changes before and after to apply a refactoring into the KDM model (measured in terms of the metric precision (P) and recall (R)) which can be formalized as: 

\textbf{H$_{0}$: $\mu_{P_{Bf}} = \mu_{P_{Af}}$ and $\mu_{R_{Bf}} = \mu_{R_{Af}}$}}
\\
\hline \cellcolor[gray]{\shadow} H$_1$ & \footnotesize{There is a significant difference in propagation of changes before and after to apply a refactoring into the KDM model (measured in terms of the metric precision (P) and recall (R)) which can be formalized as:  

\textbf{H$_{1}$: $\mu_{P_{Bf}} \neq \mu_{P_{Af}}$ and $\mu_{R_{Bf}} \neq \mu_{R_{Af}}$}}
\\
\hline
\end{tabularx}
\end{table}

%\textbf{Null hypothesis, H$_{0}$}: There is no difference in pattern recognition before and after to apply our mining affected metaclasses algorithm into the KDM model (measured in terms of the metric precision (P) and recall (R)) which can be formalized as: 

%\textbf{H$_{0}$: $\mu_{P_{Bf}} = \mu_{P_{Af}}$ and $\mu_{R_{Bf}} = \mu_{R_{Af}}$}

%\textbf{Alternative hypothesis, H$_{1}$}: There is a significant difference in pattern recognition before and after to apply our mining affected metaclasses algorithm into the KDM model (measured in terms of the metric precision (P) and recall (R)) which can be formalized as: 

%\textbf{H$_{1}$: $\mu_{P_{Bf}} \neq \mu_{P_{Af}}$ and $\mu_{R_{Bf}} \neq \mu_{R_{Af}}$}

%\textbf{Null hypothesis, H$_{0}$}: there is no difference in propagation of changes before and after to apply a refactoring into the KDM model (measured in terms of the metric precision (P) and recall (R)) which can be formalized as: 

%\textbf{H$_{0}$: $\mu_{P_{Bf}} = \mu_{P_{Af}}$ and $\mu_{R_{Bf}} = \mu_{R_{Af}}$}

%\textbf{Alternative hypothesis, H$_{1}$}: there is a significant difference in propagation of changes before and after to apply a refactoring into the KDM model (measured in terms of the metric precision (P) and recall (R)) which can be formalized as:  

%\textbf{H$_{1}$: $\mu_{P_{Bf}} \neq \mu_{P_{Af}}$ and $\mu_{R_{Bf}} \neq \mu_{R_{Af}}$}
 
There are two variables shown on each table: 'P` and 'R`. 'P` stands for Precision which is the ratio of the number of true positives retrieved/identified to the total number of irrelevant and relevant code elements retrieved/propagated. It is usually expressed as a percentage, see equation 1. R denotes Recall which is the ratio of the number of true positives retrieved/propagated to the total number of relevant code elements in the
source code. It is usually expressed as a percentage, see equation 2. 

\begin{equation}
P=\frac{True Positives}{True Positives + False Positives}
\end{equation}

\begin{equation}
R=\frac{True Positives}{True Positives + False Negatives}
\end{equation}

\subsection{Experiment Desing}

For our evaluation, we need firstly transform a system into a KDM instance to apply our approach. However, due to the scarcity of complete KDM instances in the public domain, we adopted a reverse engineering approach and generated KDM instance from one system developed in Java by using MoDisco. This system is called LabSys (Laboratory System) and it is currently used by Federal University of Tocantins (UFT) for. It is used to control the use of laboratories in the entire university. %LabSys is able to allocate time to use the classes in their respective physical spaces, treat the internal communication during laboratories reservation, such as availability reports, viability, unforeseen, acceptance of a reservation, manages the equipments of laboratories, generate reports and memos about the processes.
%
%
LabSys  was defined using the MVC architectural pattern. It contains a total of X packages and Y classes. It is composed by three layers: \texttt{model}, \texttt{view}, and \texttt{controller}. Layer \texttt{model} owns the  DTO (Data Transfer Objects) and DAOs (Data Access Objects), which is represented by \texttt{Data Package}. DTO represents domain entities such as laboratories, equipments, reservations, etc. DAO is the classes that performs the database access. Layer controller is responsible for the business rules that communicates directly with model layer. Finally, view layer is the part of the software system that performs direct interaction with the user and uses the resources of controller layer.

Currently, MoDisco only generates the KDM code package, other KDM packages are extremely important to evaluate our approach. Therefore, we have manually instantiated the followings KDM packages: Structure Package, Data Package, and Conceptual Package.

We selected three refactorings for our evaluation: \texttt{Extract Class}, \texttt{Move Class}, and \texttt{Extract Layer}. We applied each of the three refactorings to every possible location in KDM instance. It is worth to notice that all refactorings were applied completely automatically by means of our devised proof-of-concept tool. To deal with refactorings that go into infinite loops, we set three minutes timeout interval.


% More specifically, we applied the Extract ClassUnit to every class that had more than 300 LOC (Line of Code); we applied the Push Down MethodUnit to every method of a class that had a subclass that was not from a library using every such subclass as the target of the push- down; and we applied the refactoring Pull up MethodUnit to every method of a class that had a superclass that was not from a library, using every such superclass as the target of the pull-up. Then after applied all refactorings we counted whether them were successful, i.e., if the intended refactoring could be performed, and how many constraints were generated on the model and on the code side after to apply the refactorings. We also measured both software quality metrics Cohesion Amongst the Methods of a Class (CAMC) and Similarity- based Class Cohesion (SCC)4 before applying the refactoring on the KDM models and after applying the refactoring on the KDM models. Notice that in this case we actually measured these metrics in the code instead of the KDM model. This was possible as our proof-of-concept tool provides support for the generating of the code after one finishes to apply the refactorings.

\subsection{Analysis of Data and Interpretation}\label{analysis_of_data}

%This section presents the experimental findings. The analysis is divided into two subsections: (1) descriptive statistics and (2) hypothesis testing.

%\subsubsection{Descriptive Statistics:} This subsection provides descriptive statistics of the experiment datas. 

The data of the first study is found on Table X

\section{Related Work}\label{sec:related_work}
		\linespread{0.87}
		%!TEX root = /Users/rafaeldurelli/Dropbox/Artigos Elaborados/KDM propagation_2015/sbes_2015_kdm_propagation/sbes2015_kdm_propagation.tex


%In~\cite{4440135}, Enrico Biermann et al. propose to use the Eclipse Modeling Framework (EMF), a modeling and code generation framework for Eclipse applications based on structured data models. They introduce the EMF model refactoring by defining a transformation rules applied on EMF models. EMF transformation rules can be translated to corresponding graph transformation rules. If the resulting EMF model is consistent, the corresponding result graph is equivalent and can be used for validating EMF model refactoring. Authors offer a help for developer to decide which refactoring is most suitable for a given model and why, by analyzing the conflicts and dependencies of refactorings. This initiative is closed to the model driven architecture (MDA) paradigm~\cite{Kleppe:2003} since it starts from the EMF metamodel applying a transformation rules.

%In~\cite{Rui:2003} Rui, K. and Butler, apply refactoring on use case models, they propose a generic refactoring based on use case metamodel. This metamodel allows creating several categories of use case refactorings, they extend the code refactoring to define a set of use case refactorings primitive. This refactoring is very specific since it is focused only on use case model, the issue of generic refactoring is not addressed, and these works do not follow the MDA approach.

%Another work on model refactoring is proposed in~\cite{Zhang05genericand}, based on the Constraint-Specification Aspect Weaver (C-SAW), a model transformation engine which describes the binding and parameterization of strategies to specific entities in a model. Authors propose a model refactoring browser within the model transformation engine to enable the automation and customization of various refactoring methods for either generic models or domain-specific models. The transformation proposed in this work is not based on any metamodel, it is not an MDA approach.

%------------------------------------------


Westfechtel \textit{et al}.~\cite{ICSOFT2014_Winetzhammer} presented an approach for refactoring UML class diagrams and propagate the changes to behavioral models (UML Sequence diagrams), to maintain the consistency between these models/views. Unlike these authors, we aim to propagate the changes to other static views, in a cascade way. %, all of them belonging to the same family of meta-models. 
As our approach does not take into consideration behavior aspects, only static ones, we believe that both approaches are complementary to each other.

Egyed~\cite{Egyed:2006:ICC:1134285.1134339} proposed an UML-based approach similar to our second step, which is the mining and identification of model elements to be changed. In order to find those model elements, the author employs "consistency rules" between models. These rules always must keep satisfied when the models as synchronized. So, whenever an element is refactored, a broken rule is an indication that a desynchronization problem occurred, allowing the identification of model elements that must be updated to synchronize the model again. The author argue that his approach scales up to large, industrial UML models. The author employs a strategy different from ours for the identification of the points to be updated; while we rely on the comparison between the original and refactored models, he relies on the consistency rules. We believe that the problem with his approach is the insertion of another task to be performed (the specification of the consistency rules), in which new problems and errors can be inserted. Our approach is more time-consuming in terms of processing, but we believe that the recall and precision of the identification is higher.

Therefore, to be best of our knowledge our work is the first one in presenting an approach for propagating changes in KDM models in a consistent and transparent way. The most fundamental differences of other related works are: i) we consider only static models, i.e, other views of the system; ii) we work with a family of meta-models that share a consistent and homogeneously terminology (syntax) and iii) our solution is KDM-specific and iv) our approach is tool supported by means of a Eclipse plug-in which can be coupled to any KDM refactoring written in any transformation language. 

%----------------------------------------------------------------

%We are aware of only a few approaches dealing with the propagation of changes from the structural model into the behavioral model. Rosner and Bauer~\cite{murduck}  propose an approach to update model transformations in response to metamodel changes. The approach requires an ontology mapping between metamodel versions and is applied to evolve QVT-R~\cite{QVT} transformations. Similarly, Westfechtel \textit{et al}.~\cite{ICSOFT2014_Winetzhammer} presented an approach for refactoring static models (UML class diagrams, for example) and propagate the static changes to behavioral models (UML Sequence diagrams, for example), aiming to maintain the consistency between these models. Unlikely these authors, in this project our goal is to propagate the changes to other static views. It is not the purpose of our paper ensure that the refactorings maintain the observable behavioral of the system. Therefore, we believe that our approach are complementary to the proposal of the mentioned authors. 

%Recently, Egyed~\cite{Egyed:2006:ICC:1134285.1134339} proposed a very efficient approach to check for propagation (i.e. violations of consistency rules) in UML models. His approach scales up to large, industrial UML models by tracking which entities are used to check each consistency rule, and then using this information to determine which rules might be affected by a change, and only re-evaluate these rules. This work is complementary to our work: it provides a rapid means of checking consistency (which supports the first step of our approach), but does not tackle the issue of how to restore consistency.

%Research on model refactoring primarily focuses on the structural model (UML Class Diagrams). For example, in~\cite{4440135} refactoring of Ecore models is specified with graph transformation rules in the AGG environment. Differently our approach focus of model refactoring to KDM models.

%Altogether, the work presented in this paper is unique since it does not only support refactoring in KDM, but also keep all KDM levels synchronized by means of a decoupled set of ATL rules. Refactoring and propagation of changes in KDM levels are supported in an integrated/transparent way, i.e., after a refactoring changes are propagated to others KDM levels such that consistency and synchronization are preserved.

%In the line of language independent refactoring and metamodelling, Sander et al.~\cite{Tichelaar00}, study the similarities between refactorings for Smalltalk and Java, and build the FAMIX model. It provides a language-independent representation of object- oriented source code. It is an entity-relationship model that models object-oriented source code at the program entity level, with a tool to assist refactoring named MOOSE. FAMIX does not take account neither complex features in strongly typed languages, nor aspects of advanced inheritance and genericity. This approach is not really independent from language since the refactoring transformation is achieved directly on the original code. This alternative forces to implement transformers of specific code for each language. These code transformers use an approach based on text using regular expressions.





\section{Discussion}\label{sec:a_brief_discussion}		
	\linespread{0.87}
	%!TEX root = /Users/rafaeldurelli/Dropbox/Artigos Elaborados/KDM propagation_2015/sbes_2015_kdm_propagation/sbes2015_kdm_propagation.tex

\section{Discussion}\label{sec:a_brief_discussion}

The focus of our paper is the demonstration of propagation that must be performed in different static representations (views) of a given system. This means that we are not concerned with dynamic parts.
As stated earlier, previous research has demonstrated concerns about the propagation of changes when modifications are made in models. However, the largest of them are concentrated in the propagation of changes between different metamodels. As KDM is a integrated model that can be seen as a set of metamodels, where all of them are somehow connected by means of associations. This is because a certain model element is used in several places in general is referenced by its \textit{id} without having to be duplicated in multiple locations. However, as stated before, to the best of our knowledge, up to this moment, there is no research concentrated on investigating change propagation in KDM. We claim that by using our approach the modernization engineers can concentrate just on the development of the refactorings, without worrying about the change propagation, which is a time-consuming and error-prone task.


%Some partial change propagations are already developed by MoDisco\footnote{https://eclipse.org/MoDisco/} plugin. For example, when a particular model element is removed, its \textit{id} is removed from all the other places that it is used. This is considered a partial propagation, because it can, in most cases, inserting inconsistencies in the model. However, when dealing with specific refactorings it is important to keep all KDM levels synchronized 

During the elaboration of this research we realized that some propagations can also be considered as refactoring and vice versa. What characterize them is how they are used in a specific moment and not the implementation by itself. This is like having a set of refactoring that anyone can trigger anyone. When this is the case, the modernization engineer can directly apply both, unlikely propagations which clearly cannot be directly applied from the user, as it is shown in~\cite{ICSOFT2014_Winetzhammer}. This is generally the case for moving refactorings, as the moving of an element from a container to another is independent of both the container and their abstraction level. For example, suppose the existence of a class C1 belonging to a package P1. Consider also that C1 is the implementation of a business rule B1 which is inside a scenario S1 and that the package P1 is the implementation of the Scenario S1. So, C1 = B1 and P1 = S1. If the \textit{Move Class} refactoring is applied to transfer the class C1 to package P2, a natural propagation is to transfer the business rule B1 to another scenario. However, if the modernization engineer is using a modeling environment which provides a business rule view, (s)he could also have available for him(her) a moving business rule refactoring. In this case, the natural propagation would be to transfer the corresponding classes from one package to another. Therefore, we can see that in some cases there is bidirectional flow, which can be started from any point. 
%
%
%
The most important thing about this discussion is that this categorization lead us to make good designs in terms of refactorings and propagations. That is, for refactorings that fall in this category, it is very important to implement them as separated and decoupled modules which can be called directly from the user. So, all of our refactorings were implemented like that. 


\section{Conclusions and Future Work}\label{sec:conclusion}
		\linespread{0.87}
		%!TEX root = /Users/rafaeldurelli/Dropbox/Artigos Elaborados/KDM propagation_2015/sbes_2015_kdm_propagation/sbes2015_kdm_propagation.tex

The idea behind KDM is that the community starts to create parsers and tools that work exclusively over KDM instances; thus, every tool/algorithm that takes KDM as input can be considered platform and language-independent. For instance, a refactoring catalogue for KDM can be used for refactoring systems implemented in any language~\cite{IRIDurelliCatalogo}.

  A certain particularity of our approach is that the required input is two KDM instances; the original and the refactored. As the refactoring activity is not part of our approach, there is no guarantee that other modernization engineers will implement refactorings that result in two models, since an existent possibility is applying ``in place tranformations''. Therefore, a more direct application of our approach is for those engineers who have implemented ``out place refactorings'', whose output is two models.
 
Although our main focus along the paper had been on the lower-level refactorings and botton-up propagations, in our case study we decided to start an investigation on top-down propagations employing the \textit{Extract Layer} refactoring. As a result, our propagation module was able to propagate correctly even in this case, as could be shown in Table\ref{tab:prop}. However, we intend to deepen much more in this line of thought in a future work.

The main contributions are: i) a DI Algorithm to identify all KDM model elements that need to be updated when a specific refactoring is performed, ii) a propagation technique approach, and (iii) a support and preliminary infrastructure for allowing the creation of refactorings for KDM without worrying about he propagation of changes.

An important point is about the reusability of the algorithms and transformations developed in this work. All of them are strictly focused on the KDM syntax, what makes them language and platform independent. So, we could use our propagation approach during the refactoring of systems implemented in systems implemented in C++, C\#, Cobol in order to keep all their views synchronized.

 A possible future work is to integrate the proposed of Westfechtel \textit{et al}.~\cite{ICSOFT2014_Winetzhammer} with our presented approach.

\section*{Acknowledgments}
	\linespread{0.87}
	%!TEX root = /Users/rafaeldurelli/Dropbox/Artigos Elaborados/KDM propagation_2015/sbes_2015_kdm_propagation/sbes2015_kdm_propagation.tex
Rafael Serapilha Durelli would like to thank the financial support provided by FAPESP, process number 2012/05168-4. Fernando Chagas and Bruno Santos would also like to thank CNPQ.


