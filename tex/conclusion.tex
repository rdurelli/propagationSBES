%!TEX root = /Users/rafaeldurelli/Dropbox/Artigos Elaborados/KDM propagation_2015/sbes_2015_kdm_propagation/sbes2015_kdm_propagation.tex

The idea behind KDM is that the community starts to create parsers and tools that work exclusively over KDM instances; thus, every tool/algorithm that takes KDM as input can be considered platform and language-independent. For instance, a refactoring catalogue for KDM can be used for refactoring systems implemented in any language~\cite{IRIDurelliCatalogo}.

  A certain particularity of our approach is that the required input is two KDM instances; the original and the refactored. As the refactoring activity is not part of our approach, there is no guarantee that other modernization engineers will implement refactorings that result in two models, since an existent possibility is applying ``in place tranformations''. Therefore, a more direct application of our approach is for those engineers who have implemented ``out place refactorings'', whose output is two models.
 
Although our main focus along the paper had been on the lower-level refactorings and botton-up propagations, in our case study we decided to start an investigation on top-down propagations employing the \textit{Extract Layer} refactoring. As a result, our propagation module was able to propagate correctly even in this case, as could be shown in Table\ref{tab:prop}. However, we intend to deepen much more in this line of thought in a future work.

The main contributions are: i) a DI Algorithm to identify all KDM model elements that need to be updated when a specific refactoring is performed, ii) a propagation technique approach, and (iii) a support and preliminary infrastructure for allowing the creation of refactorings for KDM without worrying about he propagation of changes.

An important point is about the reusability of the algorithms and transformations developed in this work. All of them are strictly focused on the KDM syntax, what makes them language and platform independent. So, we could use our propagation approach during the refactoring of systems implemented in systems implemented in C++, C\#, Cobol in order to keep all their views synchronized.

 A possible future work is to integrate the proposed of Westfechtel \textit{et al}.~\cite{ICSOFT2014_Winetzhammer} with our presented approach.