%!TEX root = /Users/rafaeldurelli/Dropbox/Artigos Elaborados/KDM propagation_2015/sbes_2015_kdm_propagation/sbes2015_kdm_propagation.tex

In 2003 the Object Management Group (OMG) created a task force called Architecture Driven Modernization Task Force (ADMTF). It aims to analyze and evolve typical reengineering processes, formalizing them and making them to be supported by models [2]. ADM advocates the conduction of reengineering processes following the principles of Model-Driven Architecture (MDA) [22][2], i.e., all software artifacts considered along with the process are models.                                       	

According to OMG the most important artifact provided by ADM is the Knowledge Discovery Metamodel (KDM). By means of it, it is possible to represent different system abstraction levels by using its models, such as source code (Source and Code models), Actions (Action model), Architecture (Structure Model) and Business Rules (Conceptual Model). The idea behind KDM is that the community starts to create parsers and tools that work exclusively over KDM instances; thus, every tool that takes KDM as input can be considered platform and language-independent, propitiating interchange among tools. For instance, a refactoring catalogue for KDM can be used for refactoring systems implemented in different languages. 

Central to modernization processes are the refactorings. Refactorings are .....  However, most of existing model-based refactorings do not cope with propagation of the refactoring changes across other dependent abstraction levels, keeping all models synchronized [ , , , , ]

In this paper we present Propagation-Aware Refactorings (PARef), an approach for updating dependent models when specific elements are refactored.
