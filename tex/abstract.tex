%!TEX root = /Users/rafaeldurelli/Dropbox/Artigos Elaborados/KDM propagation_2015/sbes_2015_kdm_propagation/sbes2015_kdm_propagation.tex
%Architecture-Driven Modernization (ADM) is a model-driven alternative to conventional reengineering processes that relies on the Knowledge-Discovery Meta-model (KDM) as the base for the whole process. Unlike conventional meta-models, KDM is capable of putting together different system views (Code, Architecture, Business Rules, Data, Events) in an unique site and also retaining the dependencies among them. During the system life cycle, artifacts tend to change, usually these changes entail refactorings. However, as a system can be represented by several different models, a common accident that arises during refactorings is to desynchronize (inconsistent views) the models. One solution is to apply Static Change Propagation (SCP) techniques. Most of existing SCP techniques deal with propagating changes in different and external models, usually from another vendor preventing or making difficult their application in other models, like KDM. Currently there is a lack of research concentrated on investigating SCP in KDM. In this paper we present a plug-in supported KDM-specific approach for updating dependent models/views when specific elements are refactored. Our approach involve three main steps: \textit{i}) identifying diff between the refactored KDM instance and the original KDM instance (the instance before one applies a KDM refactoring), \textit{ii}) the identification of all affected KDM model elements (dependent on the refactored ones), and \textit{iii}) the propagation of changes in order to keep all the models/views synchronized. We have conducted two evaluation that shows our approach reached good accuracy and completeness levels.
%
%Architecture-Driven Modernization (ADM) is a model-driven alternative to conventional reengineering processes that relies on the Knowledge-Discovery Meta-model (KDM) as the base for the whole process. Unlike conventional meta-models, KDM is capable of putting together different views (Code, Architecture, Business Rules, etc) in an unique site and also retaining the dependencies among them. During model-driven development, a system is usually modeled by using several different models, each one representing a particular abstraction view. The general goal is to maintain these views synchronized with each other. A common accident that arises when refactoring models is to desynchronize the views and one solution is to apply Change Propagation (CP) techniques. Most of these techniques are focused on propagating changes in models that conform to different meta-models or from different vendors. Currently there is a lack of research concentrated on investigating CP in KDM. In this paper we present an approach for updating  particular abstraction views when specific KDM elements are refactored. Our approach has three steps: i) identifying diff between the refactored KDM instance and the original one, ii) the identification of all KDM model elements dependent on the refactored ones, and iii) the propagation of changes to keep all the models/views synchronized. We have conducted two evaluation that show our approach reached good accuracy and completeness when propagating the changes.
%
Architecture-Driven Modernization is a model-driven reengineering that relies on the Knowledge-Discovery Meta-model (KDM) as its main artifact. Unlike conventional meta-models, KDM is capable of grouping different  views/meta-models in an unique site and also retaining the dependencies among them. In a model-driven development, a system is modeled by using different models/views (Code, Data, Architecture, etc) and a fundamental premise is to maintain these views synchronized with each other along the process. Therefore, when models are refactored, it is important to propagate the changes throughout the other views to update them and keep them synchronized. However, most of the existing change propagation proposals are focused on models that conform to different meta-models or from different vendors, what is different from KDM. In this paper we present a change-propagation  approach to be used in the ADM context, i.e., when KDM-represented systems are refactored. Our approach has three steps: i) identifying diff between the refactored KDM instance and the original one, ii) the identification of all KDM model elements dependent on the refactored ones, and iii) the propagation of changes to keep all the models/views synchronized. We have conducted two evaluation that show our approach reached good accuracy and completeness when propagating the changes.